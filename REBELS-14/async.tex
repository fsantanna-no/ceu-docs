% SAC

The \code{par/or} statement is regarded as an \emph{orthogonal preemption 
primitive}~\cite{esterel.preemption}, because composition \code{<P>} does not 
know when and why it gets aborted by event \code{E}.

In a similar (hypothetical) construct for an asynchronous language, the 
occurrence of event \code{E} would not imply the immediate termination of 
\code{<P>}, which could continue to execute for an arbitrary amount of time.
Asynchronous formalisms assume time independence among processes and require 
explicit synchronization mechanisms.
Therefore, to achieve the desired specification, processes \code{<P>} and 
\code{<Q>} must be tweaked with synchronization commands, breaking the
orthogonality assumption.~\cite{esterel.preemption}

% SENSYS

The \code{par/or} construct is regarded as an \emph{orthogonal preemption 
primitive}~\cite{esterel.preemption} because the two sides in the composition 
need not to be tweaked with synchronization primitives or state variables in
order to affect each other.
In contrast, it is known that traditional (asynchronous) multi-threaded 
languages cannot express thread abortion 
safely~\cite{esterel.preemption,sync_async.threadsstop}.

% TR

The main limitation of the synchronous execution model is its inability to 
perform long computations requiring unbounded loops.
\emph{Asynchronous blocks} fill this gap in \CEU{} and can contain unbounded 
loops that run asynchronously with the rest of the program (referred to as the 
\emph{synchronous side}).

The program in Figure~\ref{lst:ceu:async} returns the factorial of $10$.
The loop (lines 4-11) must be inside the \code{async} (lines 2-12) as it 
contains no await statements.
The \code{return} statement (line 6) terminates the asynchronous execution 
setting the variable \code{ret} (line 2).
We use a watchdog in the enclosing \code{par/or} to cancel the computation if 
it takes longer than $10$ms (line 15).

Programming languages can be generically classified in two major execution 
models.

In the \emph{asynchronous model}, the program activities (e.g. threads and 
processes) run independently of one another as result of nondeterministic 
preemptive scheduling.
In order to coordinate at specific points, these activities require explicit 
use of synchronization primitives (e.g. mutual exclusion and message passing).

In the \emph{synchronous model}, the program activities (e.g. coroutines and 
\CEU{} trails) require explicit scheduling primitives (e.g. yield and \CEU{} 
await).
For this reason, they are inherently synchronized, as the programmer himself 
specifies when they should execute.

We use this classification to give an abstract overview of related works in 
this section, although we also comment about specific languages and systems in 
our discussion.

Note that the terms synchronous and asynchronous are somewhat ambiguous, as 
they may be used in different contexts.
The reason is that \emph{synchronous languages} require \emph{asynchronous 
primitives} (i.e. nonblocking calls), while \emph{asynchronous languages} 
require \emph{synchronous primitives} (e.g. locks and semaphores).
We use the definition of synchronous languages as found in 
\cite{rp.twelve,rp.hypothesis}.

The asynchronous model of computation can be sub-divided in how independent 
activities coordinate.
In \emph{shared memory} concurrency, communication is via global state, while 
synchronization is via mutual exclusion.
%Examples following this style are \emph{pthreads} and other multithreading 
%libraries.
In \emph{message passing}, both communication and synchronization happen via 
exchanging messages.

The default behavior of activities being independent hinders the development of 
highly synchronized applications.
As a practical evidence, we developed a simple application that blinks two LEDs 
in parallel with different frequencies%
\footnote{The complete source code and a video demos for the application can be 
found at \url{http://www.ceu-lang.org/TR/\#blink}.}.
We implemented it in \CEU{} and also in the two asynchronous styles.
For \emph{shared memory} concurrency, we used a multithreaded RTOS%
\footnote{\url{http://www.chibios.org/dokuwiki/doku.php?id=start}}, while for 
message passing, we used an \emph{occam} for Arduino~\cite{arduino.occam}.

The implementations are intentionally naive: they just spawn the activities 
that blink the LEDs in parallel.
The behavior for the asynchronous implementations of the blinking application 
is perfectly valid, as the preemptive execution model does not ensure implicit 
synchronization among activities.
We used timers in the application, but any kind of high frequency input would 
also behave nondeterministically in asynchronous systems.

Although this application can be implemented correctly with an asynchronous 
execution model, it circumvents the language style, as timers need to be 
synchronized in a single thread.
Furthermore, it is common to see similar naive blinking examples in official 
examples of asynchronous systems%
\footnote{
Example 1 in the RTOS \emph{DuinOS v0.3}:
\url{http://multiplo.org/duinos/wiki}.\\
Example 3 in the occam-based \emph{Concurrency for Arduino v20110201.1855}:
\url{http://concurrency.cc/download}.
}, suggesting that LEDs are supposed to blink synchronized.

% ONWARD

The main limitation of the synchronous execution model is its inability to 
perform long computations requiring unbounded loops.
\emph{Asynchronous blocks} fill this gap in \CEU{} and can contain unbounded 
loops that run asynchronously with the rest of the program (referred to as the 
\emph{synchronous side}).

% PLDI

On the opposite side of the spectrum of concurrency models, asynchronous 
languages for embedded systems~\cite{wsn.mantisos,arduino.occam}
assume time independence among processes and are more appropriate for 
applications with a low synchronization rate or for those involving
algorithmic-intensive problems.

Asynchronous models are also employed in real-time operating systems to provide 
response predictability, typically through prioritized 
schedulers~\cite{wsn.mantisos,wsn.oses,freertos}.
Even though \CEU ensures bounded execution for reactions, it cannot provide 
hard real-time warranties.
For instance, assigning different priorities for trails would break lock-free 
concurrency (i.e., breaking correctness is worse than breaking timeliness).
%synchronous model and
%the static analysis, which are required for

%That said, some embedded systems do require prioritized scheduling to meet 
%deadlines for critical tasks, even if it involves extra complexity to deal 
%%with synchronization issues.
Fortunately, \CEU and RTOSes are not mutually exclusive, and we can foresee a 
scenario in which multiple \CEU programs run in different RTOS threads and 
communicate asynchronously via external events, an architecture known as GALS 
(\emph{globally asynchronous--locally synchronous})~\cite{rp.gals}.

Concerning the described control-flow mechanisms, they heavily rely on 
\code{par/or} compositions, which cannot be precisely defined in asynchronous 
languages without tweaking processes with synchronization 
mechanisms~\cite{esterel.preemption}.

---
http://chplib.wordpress.com/2009/11/16/an-introduction-to-communicating-sequential-processes/

CSP allows for parallel composition of processes. Two processes composed in 
parallel run in parallel, and “a parallel combination terminates when all of 
the combined processes terminate… the best way of thinking about [this] is that 
all of the processes are allowed to terminate when they want to, and that the 
overall combination terminates when the last one does.”(R143). These semantics 
also apply to CHP’s parallel composition. CSP’s notation for composing P and Q 
in parallel: P || Q is near-identical to the CHP version: P <||> Q (they only 
differ because || is already used in the Prelude).

% CSP Hoare

specifies concurrent execution of its constituent se-
quential commands (processes). All the processes start
simultaneously, and the parallel command ends only
when they are all finished. They may not communicate
with each other by updating global variables.

% http://en.wikipedia.org/wiki/Actor_model_and_process_calculi_history

The processes were hierarchically structured using parallel composition whereas 
Actors allowed the creation of non-hierarchical execution using futures [Baker 
and Hewitt 1977]. Hierarchical parallel composition is problematical because it 
precludes the ability to create a process that outlives its creator.  Also 
message passing is the fundamental mechanism for generating parallelism in the 
Actor model; sending more messages generates the possibility of more 
parallelism.
