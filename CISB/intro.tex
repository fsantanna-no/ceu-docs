\documentclass[pdftex,12pt,a4paper]{article}

\usepackage[pdftex]{graphicx}
\usepackage{url}
\usepackage{verbatim}
\newcommand{\CEU}{\textsc{C\'{e}u}}

%\renewcommand{\chaptername}{}
%\renewcommand{\thesection}{}
%\setcounter{chapter}{-1}

\title{Safe Concurrent Abstractions for WSNs}
\date{\vspace{-5ex}}

\begin{document}

\maketitle

\section{Project Description}

Wireless sensor networks (WSNs) are composed of a large number of tiny devices 
(known as ``motes'') capable of sensing the environment and communicating among 
them.
WSNs are usually employed to continuously monitor physical phenomena in large 
or unreachable areas, such as wildfire in forests and air temperature in 
buildings.

Software for WSNs is usually developed in the C programming language, and the 
addition of a real-time operating system may extend it with preemptive and/or 
cooperative multithreading.
However, concurrency in C requires a low-level exercise related to scheduling, 
synchronizing, and the life cycle of activities (i.e. creating and destroying 
threads).

Concurrency in C also lacks safety warranties, given that they are susceptible 
to unbounded execution, race conditions and deadlocks.
Nonetheless, safety is an important aspect in WSNs, as motes have scarce 
resources, are deployed in remote locations, and must run for long periods 
without human intervention.

The programming language \CEU{} is being developed as part of the proponent's 
PhD research in PUC--Rio and is targeted at highly constrained embedded systems 
(such as WSNs), incorporating features found in dataflow and imperative 
reactive languages.

\CEU{} supports concurrent lines of execution that run in time steps and are 
allowed to share variables.
However, the synchronous and static nature of \CEU{} enables a compile time 
analysis that can enforce deterministic and memory-safe programs, offering a 
high-level and safe alternative to the predominating C based multithreaded 
systems.
The \CEU{} compiler generates code comparable to handcrafted C programs in 
terms of size and portability.

The main objective of our research during the sandwich period in Chalmers is to 
evaluate the applicability of \CEU{} in the context of Wireless Sensor Networks 
and secure communications.
The evaluation involves quantitative measures (e.g. memory usage) and
qualitative measures (e.g. ease of programming).

We believe that Wireless Sensor Networks are an ideal scenario to employ the 
\CEU{} programming language, given it is targeted at highly constrained 
embedded systems, offering fine-grained concurrency, low memory overhead, and 
safety warranties.

\section{Researcher}

\textbf{Francisco Sant'Anna} is a fourth year Ph.D. student in the Computer 
Science department at PUC-Rio. He earned his BSc (2003) and MSc (2007) degrees 
also in the Computer Science department at PUC-Rio.

The current title of his PhD thesis is ``\CEU{}: Embedded, Safe, and Reactive 
Programming'', which is expected to be concluded in September 2013.
His advisors are Prof. Roberto Ierusalimschy and Prof. Noemi Rodriguez, which 
actuate, respectively, in the field of programming languages and distributed 
systems.

\end{document}
