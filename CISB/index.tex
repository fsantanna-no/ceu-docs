\documentclass[pdftex,12pt,a4paper]{article}

\usepackage[pdftex]{graphicx}
\usepackage{url}
\usepackage{verbatim}
\newcommand{\CEU}{\textsc{C\'{e}u}}

%\renewcommand{\chaptername}{}
%\renewcommand{\thesection}{}
%\setcounter{chapter}{-1}

\begin{document}

\begin{titlepage}
\begin{center}

\textsc{\LARGE PUC--Rio}\\[1.5cm]
\textsc{\Large Research Plan}\\[0.8cm]

\newcommand{\HRule}{\rule{\linewidth}{0.5mm}}
\HRule \\[0.4cm]
{ \huge \bfseries A Reactive Programming Language for Wireless Sensor 
Networks}\\[0.4cm]
\HRule \\[1.5cm]

\begin{minipage}{0.4\textwidth}
\begin{flushleft} \large
\emph{Autor:}\\
Francisco Sant'Anna
\end{flushleft}
\end{minipage}
\begin{minipage}{0.4\textwidth}
\begin{flushright} \large
\emph{Advisors:} \\
Roberto Ierusalimschy \\
Noemi Rodriguez
\end{flushright}
\end{minipage}

\vfill
{\large \today}
\end{center}
\end{titlepage}

\tableofcontents

\newpage
\section{Abstract}
This document describes the research plan for a six-month sandwich doctorate in 
the Chalmers University in Gothenburg through the ``Ci\^encias sem Fronteiras'' 
interchange program.
The proponent, Francisco Figueiredo Goytacaz Sant'Anna, is a PhD student in 
computer science at PUC--Rio (CAPES 7).

The proposed research plan is entitled ``A Reactive Programming Language for 
Wireless Sensor Networks'' and describes how the programming language 
``\CEU{}'', being developed by the proponent during his PhD, can be used in the 
context of Wireless Sensor Networks and secure communications.

The research proposal is in the scope of Computer Sciences and Information 
Technologies, a priority area of the ``Ci\^encias sem Fronteiras'' interchange 
program.

\section{Introduction}

Wireless sensor networks (WSNs) are composed of a large number of tiny devices 
(known as ``motes'') capable of sensing the environment and communicating among 
them.
WSNs are usually employed to continuously monitor physical phenomena in large 
or unreachable areas, such as wildfire in forests and air temperature in 
buildings.
Each mote features limited processing capabilities, a short-range radio link, 
and one or more sensors (e.g. light and temperature). \cite{wsn.survey}

Software for WSNs is usually developed in the C programming language, and the 
addition of a real-time operating system may extend it with preemptive and/or 
cooperative multithreading \cite{wsn.contiki,wsn.tos}.
However, concurrency in C requires a low-level exercise related to scheduling, 
synchronizing, and the life cycle of activities (i.e. creating and destroying 
threads).

Concurrency in C also lacks safety warranties, given that they are susceptible 
to unbounded execution, race conditions and deadlocks.
Nonetheless, safety is an important aspect in WSNs, as motes have scarce 
resources, are deployed in remote locations, and must run for long periods 
without human intervention.

The programming language \CEU{} is being developed as part of the proponent's 
PhD research in PUC--Rio and is targeted at highly constrained embedded systems 
(such as WSNs), incorporating features found in dataflow and imperative 
reactive languages \cite{lustre.ieee91, esterel.design}.

\CEU{} supports concurrent lines of execution that run in time steps and are 
allowed to share variables.
However, the synchronous and static nature of \CEU{} enables a compile time 
analysis that can enforce deterministic and memory-safe programs, offering a 
high-level and safe alternative to the predominating C based multithreaded 
systems.
The \CEU{} compiler generates code comparable to handcrafted C programs in 
terms of size and portability.

Our research at PUC--Rio with \CEU{} is part of a broader project called 
``Cidades Inteligentes'' (Smart Cities), which involves 20 Brazilian 
Universities, an investment in the order of 1 million dollars, and aims to 
build an infrastructure of instrumentation, computation and communication for 
Wireless Sensor Networks.

\section{Objectives}

The main objective of our research during the sandwich period in Chalmers is to 
evaluate the applicability of \CEU{} in the context of Wireless Sensor Networks 
and secure communications.
The evaluation involves quantitative measures (e.g. memory usage) and
qualitative measures (e.g. ease of programming).

The Distributed Computing and Systems group in the Chalmers University has an 
open PhD Sandwich position in the area of ``Secure Wireless Sensor Networks''.
The research position describes programming languages technologies as one of 
its focus:
\emph{``The Security group has developed the link between two areas of computer 
science: programming languages and computer security.
The group explores security models and enforcement mechanisms based on 
programming-language technology''}.

We believe that Wireless Sensor Networks are an ideal scenario to employ the 
\CEU{} programming language, given it is targeted at highly constrained 
embedded systems, offering fine-grained concurrency, low memory overhead, and 
safety warranties.

Many number of software services are required when building a sensor network 
application.
The technique of layering services on top of each other, building functionality 
on top of functionality and, at the same time, separating them into manageable 
units are a powerful approach.

As an illustrative example, WSN applications typically run tweaked versions of 
classical distributed algorithms, such as \emph{topology discovery}, 
\emph{leader election}, \emph{broadcast}, etc \cite{ds.andrews}.
These algorithms give support for more general purpose applications and must 
run continuously with them, given the dynamicity of WSNs.

The WSN group at PUC--Rio is currently working on developing a library of a set 
of such classical algorithms.
With \CEU{}, these algorithms are easily composed with a parallel construct, 
making the main application run concurrently with the distributed algorithm 
seamlessly.
The language detects any inconsistency with the composition, such as concurrent 
access to shared variables and unreachable code.
This idea can be extended to algorithms used and/or developed by the DCS group 
in Chalmers (e.g. \cite{dcs.clusters}).

We are also planning to investigate ways to describe an approach for 
synthesizing data representations for concurrent programs.
It is part of the process to evolve the language together with the DCS group 
when identifying new requirements not currently addressed.

The DCS group in Chalmers also does research in many programming languages 
topics, which can also be explored during the PhD Sandwich 
\cite{dcs.lock_free,dcs.transactional,dcs.manycore}.
The group has a research library of concurrent constructs 
(NOBLE\footnote{http://www.cse.chalmers.se/research/group/noble/}) and a 
testbed of applications ranging from SmartGrid ones to connected car 
applications (GULLIVER\footnote{http://www.chalmers.se/hosted/gulliver-en/}).

The following list of research topics, aligned with the programming model of 
\CEU{}, was extracted from the DCS website:

\begin{itemize}
\item Investigate new techniques for achieving high parallelism and 
      fault-tolerance in distributed or parallel software.
\item Evaluate the performance of non-blocking synchronization in parallel 
      application and system software.
\item Non-blocking/fine-grain synchronization, aiming at increased parallelism, 
      fault-tolerance, avoiding convoy effects and priority inversion.
\item Enhancing performance by cooperative scheduling-and-synchronization.
\end{itemize}

\section{Methodology}

During the period of the PhD Sandwich, we intend to develop fully working WSN 
applications in \CEU{} in order to evaluate its viability as an
alternative to the current employed technologies.
We will use quantitative and qualitative metrics for the relevant aspects in
WSNs, and compare the implementations of applications in \CEU{} and other 
languages (e.g. C).

An existing work \cite{wsn.comparison} measures the performance of different 
programming languages regarding memory usage, battery
consumption, and responsiveness specifically for WSNs.
These aspects are of extreme importance, given the severe hardware constraints 
in this context.

We also consider safety an important aspect, as motes must run for long periods 
without human intervention. Finally, high expressiveness, is desired for any 
programming language in any context.

Follows the list of aspects we will evaluate during our research:

\begin{itemize}
\item Memory usage: how much applications use in terms of volatile (RAM) and 
      non-volatile (ROM) memory.
\item Battery consumption: how much energy applications consume.
\item Responsiveness: how fast applications acknowledge high-priority requests 
      (e.g. radio messages).
\item Safety: which warranties the language offers, releasing the programmer 
      from such concerns.
\item Expressiveness: how easy applications can be developed and maintained.
\end{itemize}

The first three aspects can be easily evaluated with quantitative metrics, 
while the last two require a more in-depth
analysis.

Initially, we will port to \CEU{} a set of existing applications developed in 
other languages and compare the quantitative aspects of the implementations.  
The results immediately provide feedback regarding the implementations in 
\CEU{}, which can be used to improve the language.

Then, we will focus on developing new applications with higher complexity in 
order to evaluate the qualitative aspects.
We can still use quantitative measures such as number of lines of code and 
period of development to help on the evaluation.

\section{Schedule}

Our proposal includes a six-month schedule that we intend to follow during the 
sandwich period:

\begin{table}[h]
\begin{center}
\begin{tabular}{ | l | c | c | c | c | c | c | }
\hline
    & M1 & M2 & M3 & M4 & M5 & M6 \\ \hline

Presentation of \CEU{}       & X &   & X &   & X &   \\ \hline
Development of existing apps & X & X &   &   &   &   \\ \hline
Improvements on \CEU{}       &   & x & X & x & X &   \\ \hline
Development of new apps      &   &   & X & X & X &   \\ \hline
Paper                        &   &   &   &   & X & X \\ \hline
Other activities             & x & x & x & x & x & x \\ \hline
\end{tabular}
\linebreak
{\small(\textbf{X} cells indicate high activity, while \textbf{x} cells 
indicate low activity)}
\end{center}
\end{table}

\begin{description}

\item[Months 1--2:]

During the first month, we will present the language \CEU{} to the research 
group, focusing on its applicability to WSNs and
on how it differs from existing systems.

Then, with the support of the group, we will choose existing WSN applications 
to be ported to \CEU{} in order to perform the
quantitative analysis of the implementations until the end of the second month.

\item[Months 3--4:]

With the work on existing applications, we will collect a comprehensive 
feedback to be used for improvements on the
language.

We will also present the achievements of the first two months, and discuss the 
development of new applications with
higher complexity for the next two months.

\item[Months 5--6:]

By the beginning of the fifth month, we expect to have developed some complex 
applications that can be used in a in-depth
qualitative analysis of the language.

Both the quantitative and qualitative analysis will be used in a paper to be 
written to a conference on WSNs.

\item[Months 1--6:]

We also have interest in exploring other research areas of the group related to 
programming languages and distributed
systems.

\end{description}

\section{Expected Results}

By the end of the PhD Sandwich, we expect that \CEU{} becomes a real 
alternative for the development of fully working WSN applications.
Our methodology involves a quantitative analysis that provides immediate 
feedback regarding key aspects in WSNs, such as memory usage and battery 
consumption.
Hence, we can evolve the language in a small development cycle.

We have already presented a short paper about \CEU{} in the doctoral colloquium 
of SenSys last year \cite{ceu.sensys11}, with some initial experiments that 
show competitive results in terms of memory usage and responsiveness.
With a broader usage, we will have more confidence on the measurements and 
achieve even better results.

We also expect that the ongoing research in our WSN group at PUC--Rio will take 
advantage of the interchange with the group in Chalmers and vice-versa.
Both groups share common research interests and a continuous relationship may 
arise from this first experience.
For instance, the library of distributed algorithms for WSNs being developed by 
the group at PUC--Rio is open-source and can be more easily adopted by the 
group at Chalmers through this interchange program.

We intend to write a full paper to a top conference together with the group in 
Chalmers to describe the advances of \CEU{} in the context of WSNs:

\begin{itemize}
\item ACM/IEEE Conference on Information Processing in Sensor Networks (IPSN).  
      Submission deadline expected to October 2012.
\item ACM Conference on Embedded Networked Sensor Systems (SenSys).
      Submission deadline expected to April 2013.
\end{itemize}

Another outcome of our research could be the adoption of \CEU{} in new projects 
in the DCS group in Chalmers, which need not be directly related to research in 
programming languages.

\section{Researchers}

\begin{description}

\item[PUC--Rio (proponent):]
\hspace{1mm}

\textbf{Francisco Sant'Anna} is a third year Ph.D. student in the Computer 
Science department at PUC-Rio. He earned his BSc (2003) and MSc (2007) degrees 
also in the Computer Science department at PUC-Rio.

The current title of his PhD thesis is ``\CEU{}: Embedded, Safe, and Reactive 
Programming'', which is expected to be concluded in September 2013.
His advisors are Prof. Roberto Ierusalimschy and Prof. Noemi Rodriguez, which 
actuate, respectively, in the field of programming languages and distributed 
systems.

\textbf{Prof. Roberto Ierusalimschy} is an associate professor of informatics 
at PUC-Rio (Pontifical University in Rio de Janeiro).
He is the leading architect of the Lua programming language and the author of 
Programming in Lua.\cite{lua.pil}

His research activities currently focus on programming languages, mainly 
alternative languages such as scripting languages and domain-specific 
languages.

\textbf{Prof. Noemi Rodriguez} is an associate professor of informatics at 
PUC--Rio.
Her research interests include the area of concurrent and distributed 
programming, with an emphasis on the role of programming languages in this 
context.
She also leads the Wireless Sensor Networks research group at PUC--Rio.

\item[Chalmers University (host):]
\hspace{1mm}

\textbf{Prof. Philippas Tsigas} is a professor in the Department of Computing 
Science and Engineering at Chalmers University.
He is the co-leader of the Distributed Computing and Systems Research Group.
His research interests center on distributed/parallel computing and systems and 
information visualization in general.

Homepage:
\url{http://www.cse.chalmers.se/~tsigas/index.html}

\end{description}

\bibliographystyle{acm}
\bibliography{other,my}

\end{document}

