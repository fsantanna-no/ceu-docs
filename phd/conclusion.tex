% TTT: corroborar results dos outros trabalhos

We presented \CEU, a system-level programming language targeting 
control-intensive WSN applications.
\CEU is based on a synchronous core that combines parallel compositions with 
standard imperative primitives, such as sequences, loops and assignments.
%
Our work has three main contributions:
%
\begin{itemize}
%
\item A resource-efficient synchronous language that can express control 
      specifications concisely.
%
\item The stack-based execution policy for internal events as a powerful 
      broadcast communication mechanism.
%
\item A wide set of compile-time safety guarantees for concurrent programs that 
      are still allowed to share memory and access the underlying platform in 
``raw $C$''.
%
\end{itemize}

%
%The static analysis algorithm is actually quite simple and effective .
 %and design decisions that made it possible are quite simple:
%, and obvious if we take for granted the design decisions behind (e.g.
%pushing towards
%The series of decision, discrete execution, uniqueness of events

We argue that the dictated safety mindset of our design does not lead to a 
tedious and bureaucratic programming experience.
%
In fact, the proposed safety analysis actually depends on control information 
that can only be inferred based on high-level control-flow mechanisms (which 
results in more compact implementations).
%
Furthermore, \CEU embraces practical aspects for the context of WSNs, providing 
seamless integration with $C$ and a convenient syntax for timers.
%
%This relationship between safety and expressiveness is two-way, and
%also rely on the safety guarantees to be trustworthy.

As far as we know, \CEU is the first language with stack-based internal events, 
which allows to build rich control mechanisms on top of it, such as a limited 
form of subroutines and exception handling.
%
In particular, \CEU's subroutines compose well with the other control 
primitives and are safe, with guaranteed bounded execution and memory 
consumption.

We presented two complete demos to show how typical patterns in WSNs such as 
sampling, timeout and concurrency can be easily implemented.
They also explore parallel compositions for specifying complementary activities 
in separate.
Communication among activities can either use internal events or safe access to 
global variables.

Our evaluation compares several implementations of widely adopted WSN protocols 
in \CEU to \nesc, showing a considerable reduction in code size with a small 
increase in resource usage.
%
On the way to a more in-depth qualitative approach, such as evaluating the 
leraning curve of the language, we have been teaching \CEU as an alternative to 
\nesc in hands-on WSN courses in a high school for the past two years (and also 
in two universities in short courses).
Our experience shows that students are capable of implementing an 
\emph{SRP}-like routing protocol in \CEU in a couple of weeks.

We presented a formal semantics for the control aspects of \CEU and discussed 
how they are implemented in $C$.
%
The resource-efficient implementation of \CEU is suitable for constrained 
sensor nodes and imposes a small memory overhead in comparison to handcrafted 
event-driven code.

We believe that the strong position in favor of shared-memory concurrency is 
also a contribution of the thesis:
%
first because although synchronous languages emerged in the early eighties, we 
are not aware of derived work allegedly addressing this issue;
%
second because the current trend in the programming-languages community is 
towards the adoption of more pure functional languages and message-passing 
concurrency to get rid of shared memory, which is in the opposite direction of 
\CEU.

%\subsection{Ongoing work}
