In this section, we explore the stacked execution for internal events in \CEU, 
demonstrating how it enables advanced control-flow mechanisms in the language 
without requiring new primitives.
We first describe a restricted form of subroutines that is used as the basis 
for the more elaborate mechanisms.

Although the described mechanisms involve thoughtful techniques, they can be 
easily abstracted with compile-time macros taking advantage of the structured 
style of \CEU%
\footnote{Our programs in \CEU make extensive use of the \emph{m4} 
preprocessor.}.
As an exception, the \DOFIN construct to be presented in 
Section~\ref{sec.adv.fin} makes slight global additions to the program tree and 
requires a dedicated syntax.

\subsection{Subroutines}
\label{sec.adv.sub}

Internal events bring support for a limited form of subroutines.
In the example that follows, we define a function \code{inc} that increments 
the value passed as reference.
A trail in parallel calls this function in reaction to external event \code{A}:
 
{\small
\begin{verbatim}
 1:  event int* inc;    // function `inc' receives an int
 2:  par/or do
 3:      loop do        // function definitions are loops
 4:          int* p = await inc; // that await the event
 5:          *p = *p + 1;        // to execute the body
 6:      end                     // and await again
 7:  with
 8:      int v = 1;
 9:      await A;
10:      emit inc(&v);           // call `inc'
11:      _assert(v==2);          // assert after return
12:  end
\end{verbatim}
}

A subroutine is represented as a loop that awaits an identifying event 
(\code{await inc}, in line 4).
A subroutine is called in a parallel trail through an emit on the corresponding 
event (\code{emit inc}, in line 10).
The parameter of a subroutine is the type of its corresponding event 
(\code{event int* inc}, in line 1).

In the example, the trails start awaiting events \code{inc} and \code{A}.
Once the input event \code{A} occurs, the second trail awakes and invokes 
\code{emit~inc(\&v)} to ``call'' subroutine \code{f}.
Given the stacked execution for internal events, the calling trail pauses and 
the subroutine awakes.
The subroutine increments the parameter, loops, and awaits to be called again.
Finally, the calling trail resumes and passes the assertion test, terminating 
the program.

This form of subroutines has some significant limitations:

\begin{description}
\item[\emph{Single calling}:] Further calls to a subroutine in a reaction chain 
have no effect.
Suppose that after the assertion test in line 11, the trail tries to call the 
function again.
Given that awaking $await$ statements require them to be $awaiting$ before a 
reaction chain starts, the second call is ignored.
Remember that events are not buffered in \CEU.

\item[\emph{No recursion}:] Recursive calls to a subroutine also have no 
effect.
Suppose that after the increment in line 5, the subroutine tries to call 
itself.
As the trail cannot be awaiting itself while running, the recursive call is 
ignored.

\item[\emph{Single instance}:] Calls to a running subroutine also have no 
effect.
Suppose that after the increment in line 5, the subroutine awaits an event 
(e.g. an external event \code{B}).
While this event does not occur, the subroutine is hanged and cannot serve 
other requests (even in other reaction chains).

\item[\emph{No concurrency}:] If two trails in parallel try to call the same 
subroutine passing a parameter, the static analysis complains about 
nondeterminism.
Considering the formal semantics, an $emit$ needs to be translated to 
$mem;emit(e)$ in order to describe parameter passing (where $mem$ accesses 
$e$).
This way, the concurrent $mem$ operations would be detected as 
nondeterministic.
\end{description}

\vspace{5pt}
\CEU provides no support for standard functions for a number of reasons:
\begin{itemize}
\item The interaction with other \CEU control primitives is not obvious (e.g., 
executing an $await$ or a $par/or$ inside a function).
\item They would still be restricted in some ways given the embedded context 
(e.g.  no recursion or closures).
\item Programs can always recur to $C$ when absolutely necessary.
%\item A dedicated primitive would behave just as described, being a matter of 
%syntactic sugar.
\end{itemize}

Regardless of the limitations, this form of subroutines is widely adopted in 
\CEU programs, given that they were designed to work with the other control 
mechanisms.
Keep in mind that the typical reactive organization of programs (awaiting an 
external stimulus, reacting to it, and going back to awaiting) does not demand 
unrestricted subroutines.
In Section~\ref{sec.adv.frp}, we show that we can even take advantage of 
non-recursive subroutines to properly describe mutual dependency among trails 
in parallel.

\subsection{Finalization blocks}
\label{sec.adv.fin}

Finalization blocks (as found in $Java$ and $C\#$) are often useful to handle 
dynamic resource allocation in a structured way.
As an example, the naive program in \CEU that follows allocates a block of 
memory and uses it across reactions to events before freeing it:

{\small
\begin{verbatim}
    input void A,F;
    par/or do
        tp* ptr = _malloc(...);
        ... // use `ptr'
        await A;
        ... // use `ptr'
        _free(ptr);
    with
        await F;
    end
    ...     // program continues
\end{verbatim}
}

In the program, if event \code{F} occurs before \code{A}, the \code{par/or} 
composition terminates and does not free the allocated memory, leading to a 
leak.

\CEU provides a \DOFIN construct to ensure the execution of a block of code to 
safely release resources.
The previous example can be rewritten as the code in the left side of
Figure~\ref{lst:finally}, which forces the execution of the finalization block 
after the \FIN keyword, even when the outer \code{par/or} terminates.

\DOFIN constructs do not add any complexity to the semantics of \CEU, relying 
only on the set of primitives already presented in 
Section~\ref{sec.sem.syntax}.
% TODO: RW
For instance, the example is translated at compile time into the code shown in 
the right side of the figure, as follows:

\begin{enumerate}
\item A unique global internal event \code{fin} is declared.%
\footnote{Each \DOFIN is associated to an unique event (e.g.,  \code{fin\_1}, 
\code{fin\_2}, etc.).}
\item The \DOFIN is converted into a \code{par/and}.
\item The first \code{par/and} trail emits \code{fin} on termination to invoke 
the finalization block.
\item The second \code{par/and} trail (the finalization block) awaits 
\code{fin} to start executing.
\item All trails that terminate a \code{par/or} or escape a \code{loop} emit 
\code{fin} to also invoke the finalization block.
\end{enumerate}

We opted for a dedicated syntax given that the transformation is not 
self-contained, affecting the global structure of programs.

The cases that follow illustrate the precise behavior of finalization blocks 
when a third trail in parallel encloses a \DOFIN construct and kills it:

\begin{itemize}
\item \emph{3rd trail terminates before the \DOFIN starts to execute.}
In this case, 3rd trail emits the corresponding \code{fin}, which is not yet 
being awaited for, and the finalization block does not execute.
\item \emph{3rd trail terminates while the \DOFIN is blocked.}
In this case, the resource has been acquired but not released.
The corresponding \code{fin} is emitted and holds 3rd trail to awake the 
finalization block, which safely releases the resource before resuming the 
terminating trail.
\item \emph{3rd trail terminates concurrently with the \DOFIN.} (Suppose they 
react to the same event.)
In this case, both trails emit \code{fin}, executing the finalization block 
only once, as expected.
\end{itemize}

\begin{figure}[t]
%\rule{14cm}{0.37pt}
{\small
\begin{minipage}[t]{0.45\linewidth}
\begin{alltt}
  input void A,F;

  par/or do
    \textbf{do}
      \_t* ptr = \_malloc();
      ... // use `ptr'
      await A;
      ... // use `ptr'

    \textbf{finally}

      \_free(ptr);
    \textbf{end}
  with
    await F;

  end
\end{alltt}
\end{minipage}
%
\hspace{0.5cm}
%
\begin{minipage}[t]{0.45\linewidth}
\begin{alltt}
input void A,F;
\textbf{event void fin;}      (1)
par/or do
  \textbf{par/and do}          (2)
    \_t* ptr = \_malloc();
    ... // use `ptr'
    await A;
    ... // use `ptr'
    \textbf{emit fin;}        (3)
  \textbf{with}
    \textbf{await fin};       (4)
    \_free(ptr);
  \textbf{end}
with
  await F;
  \textbf{emit fin;}          (5)
end
\end{alltt}
\end{minipage}

\caption{ \DOFIN code and corresponding translation.
\label{lst:finally}
}
}
\end{figure}

%\TODO: limitation: `do´ must await

\DOFIN constructs have the restriction that finalization code cannot await 
events, otherwise they would be killed by the terminating trail before 
releasing the acquired resources.
However, releasing resources does not typically involve awaiting.

\subsection{Exception handling}
\label{sec.adv.excpt}

Exception handling can be provided by specialized programming language 
constructs (e.g., \code{try-catch} blocks in Java), but also with techniques 
using standard control-flow primitives (e.g., \code{setjmp/longjmp} in $C$).
\CEU can naturally express different forms of exception handling without a 
specific construct.

As an illustrative example, suppose an external entity periodically writes to a 
log file and notifies the program through the event \code{ENTRY}, which carries 
the number of available characters to read.
We start with the simple and straightforward specification of 
Figure~\ref{lst:excpt1}.
The normal flow is to open the file and wait in a loop for \code{ENTRY} 
occurrences.
We use a finalization block to safely close the file in the case of abrupt 
terminations, as discussed in previous section.
The low-level file operations \code{open} and \code{read} are defined as 
internal events working as subroutines.
%A \code{read} operation does not await external events, hence, the program 
%loops and awaits event \code{ENTRY} continuously, reacting to all log writes.

\begin{figure}[t]
%\rule{14cm}{0.37pt}
{\small
\begin{verbatim}
 // DECLARATIONS
 input int START;  // start handling the log
 input int ENTRY;  // new log entry
 _FILE*    f;      // holds a reference to the log
 char[10] buf;     // holds the current log entry
 event char* open; // opens filename into `f'
 event int read;   // reads a number of bytes into `buf'
 event int excpt;  // callback event for exceptions

 // NORMAL FLOW
 await START;
 do
    emit open("log.txt");
    loop do
       int n = await ENTRY;
       emit read(n);              // reads into `buf'
       _printf("log: %s\n", buf); // handles log string
    end
 finally
    if f != _NULL then
       _fclose(f);
    end
 end
\end{verbatim}
}%
\caption{ Program to handle log entries.
\label{lst:excpt1}
}
\end{figure}

The operations that perform the actual low-level system calls are placed in 
parallel and may emit exceptions through event \code{excpt}, as 
Figure~\ref{lst:excpt2} shows.

\begin{figure}[t]
{\small
\begin{verbatim}
 // DECLARATIONS (as in previous code)
 par/or do
     // NORMAL FLOW (as in previous code)
 with
     loop do     // OPEN subroutine
         char* filename = await open;
         f = _open(filename);
         if f == _NULL then
             emit excpt(1);  // 1 = open exception
         end
     end
 with
     loop do     // READ subroutine
         int n = await read;
         if (n > 10) || (_read(f,buf,n) != n) then
             emit excpt(2);  // 2 = read exception
         end
     end
 end
\end{verbatim}
}%
\caption{ Low-level operations are placed in parallel.
\label{lst:excpt2}
}
\end{figure}

To handle exceptions, we enclose the normal flow with another \code{par/or} to 
terminate it on any exception thrown by file operations:

{\small
\begin{verbatim}
 // DECLARATIONS
 par/or do
     par/or do
         // NORMAL FLOW
     with
         await excpt;    // catch exceptions
     end
 with
     // OPERATIONS       // throw exceptions
 end
\end{verbatim}
}

To illustrate an exception, suppose the normal flow tries to read a string and 
fails.
The program behaves as follows (with the stack in emphasis):

{\small
\begin{enumerate}
\setlength{\itemsep}{0pt}
\item Normal flow invokes the read operation (\code{emit read}) and pauses;\\
    \emph{stack: [norm]}
\item Read operation awakes, throws an exception (\code{emit excpt}), and 
    pauses;\\
    \emph{stack: [norm, read]}
\item Exception handler (\code{await excpt}) awakes, invokes the \FIN (through 
    implicit \code{emit fin}), and pauses;\\
    \emph{stack: [norm, read, hdlr]}
\item The \FIN block executes, closes the file, and terminates;\\
    \emph{stack: [norm, read, hdlr]}
\item The exception continuation terminates the \code{par/or}, cancelling all 
remaining stacked continuations.\\
    \emph{stack: []}
\end{enumerate}
}

This mechanism for exceptions can also support resumption if the handler does 
not terminate its surrounding \code{par/or}.
For instance, the new handler of Figure~\ref{lst:excpt3} waits for exceptions 
in a loop and recovers from each type of exception.

\begin{figure}[t]
{\small
\begin{verbatim}
 ...
     par/or do
         // NORMAL FLOW
     with
         loop do
             int err = await excpt;  // catch exceptions
             if err == 1 then        // open exception
                 f = <creates a new file>
             else/if err == 2 then   // read exception
                 buf = <assigns a default string>
             end
         end
     end
 ...
\end{verbatim}
}%
\caption{ Exception handling with resumption.
\label{lst:excpt3}
}
\end{figure}

Now, step 3 in the previous execution trace would not fire the \FIN block, but 
instead, assign a default string to \code{buf}, loop and await the next 
exception.
Then, the exception continuation would loop and await further file operations.
In the end, the read operation would resume as if no exceptions had occurred.

Note that throughout the example, the normal flow of Figure~\ref{lst:excpt1} 
remained unchanged, with all machinery to handle exceptions placed around it.
Also, although we use globals in the example (\code{f} and \code{buf}), 
remember that they are guaranteed to be safely accessed.
%The only modification was actually a bug removal, as we included the 
%\code{do-finally} block to ensure closing the file safely.

In terms of memory usage, switching from the original normal flow (without 
exception throws) to the last example (with recovery) incurred extra 450 bytes 
of ROM and 24 bytes of RAM.

The presented approach for exceptions has the limitation that a file operation 
cannot be called twice within a reaction chain and that exception handlers 
cannot await other events, which are related to the single-call and 
single-instance property of subroutines in \CEU.

\subsection{Dataflow programming}
\label{sec.adv.frp}

Reactive dataflow programming \cite{frp.survey} provides a declarative style to 
express dependency relationships among data.
Mutual dependency is a known issue in dataflow languages, requiring the 
explicit placement of a specific delay operator to avoid runtime
cycles~\cite{frtime.embedding,luagravity.sblp}.
This solution is somewhat \emph{ad hoc} and splits an internal dependency 
problem across two reactions to the environment.
%It also requires the mutual dependency to eventually converge to a value so 
%that variables do not affect each other forever.

\CEU can naturally express safe mutual dependencies, making it impossible to 
implement recursive definitions (as shown in Section~\ref{sec.safety.bounded}).
For instance, the program in Figure~\ref{lst:ceu:frp:2} applies the temperature 
conversion formula between Celsius and Fahrenheit, so that whenever the value 
in one unit is set, the other is automatically recalculated (a problem proposed 
in~\cite{frp.survey}).

\begin{figure}[t]
%\rule{14cm}{0.37pt}
{\small
\begin{verbatim}
 1:   event int TC, TF;
 2:   int tc, tf;
 3:   event int tc_evt, tf_evt;
 4:   par/or do
 5:      loop do                // 1st trail
 6:         tc = await tc_evt;
 7:         emit tf_evt(9 * tc / 5 + 32);
 8:      end
 9:   with
10:      loop do                // 2nd trail
11:         tf = await tf_evt;
12:         emit tc_evt(5 * (tf-32) / 9);
13:      end
14:   with
15:      loop do
16:         int v = await TC;   // 3rd trail
17:         emit tc_evt(v);
18:         ...   // use `tc' or `tf'
19:      end
20:   with
21:      loop do
22:         int v = await TF;   // 4th trail
23:         emit tf_evt(v);
24:         ...   // use `tc' or `tf'
25:      end
26:   end
\end{verbatim}
}%
\caption{ A dataflow program with mutual dependency.
\label{lst:ceu:frp:2}
}
\end{figure}

We first define the external events that signal changes, the variables to hold 
the temperatures, and corresponding internal events (lines 1-3).
Any change to a variable in the program must be signalled by an emit on the 
corresponding internal event so that dependent variables can react.
Then, we create two trails to await for internal changes and update the 
dependency relations among the temperatures (lines 5-8 and 10-13).
For instance, the first trail is a \code{loop} that waits for changes on 
\code{tc\_evt} (line 6) and signals the conversion formula to \code{tf\_evt} 
(line 7).
The behavior for the second trail that awaits \code{tf\_evt} (lines 10-13) is 
analogous.
The third and fourth trails (lines 15-19 and 21-25) await external updates in
loop to notify the internal changes;
The program behaves as follows (with the stack in emphasis):

{\small
\begin{enumerate}
\setlength{\itemsep}{0pt}
\item 1st and 2nd trail await \code{tc\_evt} and \code{tf\_evt};\\
    \emph{stack: []}
\item If \code{TC} occurs, 3rd trail signals a change to \code{tc\_evt} and 
    pauses;\\
    \emph{stack: [3rd]}
\item 1st trail awakes, sets \code{tc=0}, emits \code{tf\_evt}, and pauses;\\
    \emph{stack: [3rd,1st]}
\item 2nd trail awakes, sets \code{tf=32}, emits \code{tc\_evt}, and pauses;\\
    \emph{stack: [3rd,1st,2nd]}
\item no trails are awaiting \code{tc\_evt} (1st trail is paused), so 2nd trail 
    (on top of the stack) resumes, loops, and awaits \code{tf\_evt} again;\\
    \emph{stack: [3rd,1st]}
\item 1st trail resumes, loops, and awaits \code{tc\_evt} again;\\
    \emph{stack: [3rd]}
\item 3rd trail resumes \emph{with all dependencies resolved} and awaits the 
    next external change;\\
    \emph{stack: []}
\item ... (analogous behavior for further external occurrences)
\end{enumerate}
}

The complexity of the solution is disproportionate to the problem it solves, 
but illustrates the circular dependency issue (similar examples appear in other 
references~\cite{frp.survey,frtime.embedding}).
The bottom line is that dataflow techniques permit that complex dependency 
patterns are handled internally, providing well-defined entry points to 
application programmers (i.e. they would be required to write only the 3rd and 
4th trails in the example).

%Altough xxx, no support for dynamic reconfiguration given the static nature of
