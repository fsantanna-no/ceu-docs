%\begin{document}

The disciplined synchronous execution of \CEU, together with broadcast 
communication and stacked execution for internal events, may raise doubts about 
the precise execution of programs.
%
In this chapter, we introduce a reduced syntax of \CEU and propose an 
operational semantics in order to formally describe the language, eliminating 
imprecisions with regard to how a program reacts to an external event.
%
For the sake of simplicity, we focus on the control aspects of the language, 
leaving out side effects and $C$ calls (which behave like in any conventional 
imperative language).

%The full syntax and semantic rules can be obtained in 
%Appendix~\ref{app.formal}.

%In section~\ref{sec.formal.ceu}, we show how to translate \CEU statements into

\section{Abstract syntax}
\label{sec.sem.syntax}

\begin{figure}[h]
%\rule{8.5cm}{0.37pt}
{\small
\begin{verbatim}
                                   // primary expressions
  p ::= mem(id)                    (any memory access to `id')
      | await(id)                  (await event `id')
      | emit(id)                   (emit event `id')
      | break                      (loop escape)
                                   // compound expressions
      | if mem(id) then p else p   (conditional)
      | p ; p                      (sequence)
      | loop p                     (repetition)
      | p and p                    (par/and)
      | p or p                     (par/or)
      | fin p                      (finalization)
                                   // derived by semantic rules
      | awaiting(id,n)             (awaiting `id' since sequence number `n')
      | emitting(n)                (emitting on stack level `n')
      | p @ loop p                 (unwinded loop)
\end{verbatim}
}%
\caption{
    Reduced syntax of \CEU.
\label{lst.formal.syntax}
}
\end{figure}

Figure~\ref{lst.formal.syntax} shows the BNF-like syntax for a subset of \CEU 
that is sufficient to describe all semantic peculiarities of the language.
%
The $mem(id)$ primitive represents all accesses, assignments, and $C$ function 
calls that affect a memory location identified by $id$.
As the challenging parts of \CEU reside on its control structures, we are not 
concerned here with a precise semantics for side effects, but only with their 
occurrences in programs.
%In accordance with the zero-delay hypothesis of \CEU, $mem$ expressions are 
%considered to be atomic and instantaneous
%
The special notation $nop$ is used to represent an innocuous $mem$ expression 
(it can be though as a synonym for $mem(\epsilon)$, where $\epsilon$ is an 
unused identifier).
%
Except for the $fin$ and semantic-derived expressions, which are discussed 
further, the other expressions map to their counterparts in the concrete 
language in Figure~\ref{lst.syntax}.
%
Note that $mem$ expressions cannot share identifiers with $await$/$emit$ 
expressions.

\section{Operational semantics}
\label{sec.sem}

The core of our semantics is a relation that, given a sequence number $n$ 
identifying the current reaction chain, maps a program $p$ and a stack of 
events $S$ in a single step to a modified program and stack:
%
$$
\LL S, p \RR
    \xrightarrow[~~n~~]{}
\LL S', p' \RR
$$
%
where
%
\begin{align*}
S, S' &\in id^*
    &&(sequence~of~event~identifiers: [id_{top}, ..., id_1]) \\
p, p' &\in P
    && (as~described~in~Figure~\ref{lst.formal.syntax}) \\
n     &\in \mathds{N}
    && (univocally~identifies~a~reaction~chain)
\end{align*}

At the beginning of a reaction chain, the stack is initialized with the 
occurring external event $ext$ ($S=[ext]$), but $emit$ expressions can push new 
events on top of it (we discuss how they are popped further).

%Therefore, there is a single order of execution for side effects (represented 
%by \code{mem} operations), thus, reaction chains are always deterministic.
%
% TODO
%\footnote{We could extend the semantics to describe the full execution of a 
%program by holding new incoming external events in a queue and processing them 
%in consecutive reaction chains that never overlap.}
%
%The semantics requires an explicit stack to properly nest the emission of 
%internal events.
%to prevent that $awaiting$ expressions awake in the same reaction they are 
%reached.

We describe this relation with a set of \emph{small-step} structural semantics 
rules, which are built in such a way that at most one transition is possible at 
any time, resulting in deterministic reaction chains.
%
The transitions rules for the primary expressions are as follows:
%
{ \setlength{\jot}{20pt}
\begin{align*}
\LL S,\1await(id) \RR &\ST
\LL S,\1awaiting(id,n) \RR
    & \textbf{(await)}      \\
%%%
\LL id:S,\1awaiting(id,m) \RR &\ST
\LL id:S,\1nop \RR, \2m<n
    & \textbf{(awaiting)}   \\
%%%
\LL S,\1emit(id) \RR &\ST
\LL id:S,\1emitting(|S|) \RR
    & \textbf{(emit)}       \\
%%%
\LL S,\1emitting(|S|) \RR &\ST
\LL S,\1nop \RR
    & \textbf{(emitting)}
\end{align*}
}

An $await$ is simply transformed into an $awaiting$ that remembers the current 
external sequence number $n$ (rule \textbf{await}).
An $awaiting$ can only transit to a $nop$ (rule \textbf{awaiting}) if its 
referred event $id$ matches the top of the stack and its sequence number is 
smaller than the current one ($m<n$).
%
%Remember that in \CEU, the \code{await} statement returns the value associated 
%with the corresponding event: the yielded $mem$ represents the operation to 
%query that value.
%
An $emit$ transits to an $emitting$ holding the current stack level ($|S|$ 
stands for the stack length), and pushing the referred event on the stack (in 
rule \textbf{emit}).
With the new stack level $|S|+1$, the $emitting(|S|)$ itself cannot transit, as 
rule \textbf{emitting} expects its parameter to match the current stack level.
This trick provides the desired stack-based semantics for internal events.

Proceeding to compound expressions, the rules for conditionals and sequences 
are straightforward:

{ \setlength{\jot}{20pt}
\begin{eqnarray*}
& \frac
    { \DS val(id,n) \neq 0 }
%   -----------------------------------------------------------
    { \DS \LL S, (if~mem(id)~then~p~else~q) \RR \ST
          \LL S, p \RR }
    & \textbf{(if-true)}       \\
%%%
& \frac
    { \DS val(id,n) = 0 }
%   -----------------------------------------------------------
    { \DS \LL S, (if~mem(id)~then~p~else~q) \RR \ST
          \LL S, q \RR }
    & \textbf{(if-false)}       \\
%%%
& \frac
    {\DS \LL S,p \RR \ST \LL S',p' \RR }
%   -----------------------------------------------------------
    {\DS \LL S, (p~;~q) \RR \ST \LL S', (p'~;~q) \RR }
    & \textbf{(seq-adv)}      \\
%%%
& \LL S, (mem(id)~;~q) \RR \ST  \LL S, q \RR
    & \textbf{(seq-nop)}      \\
%%%
& \LL S, (break~;~q) \RR \ST \LL S, break \RR
    & \textbf{(seq-brk)}
\end{eqnarray*}
}

Given that our semantics focuses on control, rules \textbf{if-true} and 
\textbf{if-false} are the only to query $mem$ expressions.
%
The ``magical'' function $val$ receives the memory identifier and current 
reaction sequence number, returning the current memory value.
%
Although the value is arbitrary, it is unique in a reaction chain, because a 
given expression can execute only once within it (remember that $loops$ must 
contain $awaits$ which, from rule \textbf{await}, cannot awake in the same 
reaction they are reached).
%For all other rules, we omit these values (e.g., \textbf{seq-nop}).

The rules for loops are analogous to sequences, but use \code{`@'} as 
separators to properly bind breaks to their enclosing loops:

{ \setlength{\jot}{20pt}
\begin{eqnarray*}
& \LL S, (loop~p) \RR \ST \LL S, (p~@~loop~p) \RR
    & \textbf{(loop-expd)}       \\
%%%
& \frac
    {\DS \LL S,p \RR \ST \LL S',p' \RR }
% -----------------------------------------------------------
    {\DS \LL S, (p~@~loop~q) \RR \ST \LL S', (p'~@~loop~q) \RR }
    & \textbf{(loop-adv)}    \\
%%%
& \LL S, (mem(id)~@~loop~p) \RR \ST \LL S, loop~p \RR
    & \textbf{(loop-nop)}    \\
%%%
& \LL S, (break~@~loop~p) \RR \ST \LL S, nop \RR
    & \textbf{(loop-brk)}
\end{eqnarray*}
}

When a program first encounters a $loop$, it first expands its body in sequence 
with itself (rule \textbf{loop-expd}).
Rules \textbf{loop-adv} and \textbf{loop-nop} are similar to rules 
\textbf{seq-adv} and \textbf{seq-nop}, advancing the loop until they reach a 
$mem(id)$.
However, what follows the loop is the loop itself (rule \textbf{loop-nop}).
Note that if we used \code{`;'} as a separator in loops, rules 
\textbf{loop-brk} and \textbf{seq-brk} would conflict.
%
Rule \textbf{loop-brk} escapes the enclosing loop, transforming everything into 
a $nop$.
%Rule \textbf{loop-brk} escapes the enclosing loop, transforming everything 
%into a $clear(p)$.
%We cannot simply transform the loop into a $nop$ because its body may be a 
%parallel composition containing finalization blocks.

The rules for parallel $and$ compositions force transitions on the left branch 
$p$ to occur before transitions on the right branch $q$ (rules 
\textbf{and-adv1} and \textbf{and-adv2}).
Then, if one of the sides terminate, the composition is simply substituted by 
the other side (rules \textbf{and-nop1} and \textbf{and-nop2}):

{ \setlength{\jot}{20pt}
\begin{eqnarray*}
& \frac
    {\DS \LL S,p \RR \ST \LL S',p' \RR }
%   -----------------------------------------------------------
    {\DS \LL S, (p~and~q) \RR \ST \LL S', (p'~and~q) \RR }
    & \textbf{(and-adv1)}      \\
%%%
& \frac
    {\DS isBlocked(n,S,p) \1,\2 \LL S,q \RR \ST \LL S',q' \RR }
%   -----------------------------------------------------------
    {\DS \LL S, (p~and~q) \RR \ST \LL S', (p~and~q') \RR }
    & \textbf{(and-adv2)}      \\
%%%
& \LL S, (mem(id)~and~q) \RR \ST \LL S, q \RR
    & \textbf{(and-nop1)}   \\
%%%
& \LL S, (p~and~mem(id)) \RR \ST \LL S, p \RR
    & \textbf{(and-nop2)}   \\
%%%
& \LL S, (break~and~q) \RR \ST \LL S, (clear(q)~;~break) \RR
    & \textbf{(and-brk1)}   \\
%%%
& \frac
    {\DS isBlocked(n,S,p) }
%   -----------------------------------------------------------
    {\DS \LL S, (p~and~break) \RR \ST \LL S, (clear(p)~;~break) \RR }
    & \textbf{(and-brk2)}   \\
\end{eqnarray*}
}

The deterministic behavior of the semantics relies on the \emph{isBlocked} 
predicate, defined in Figure~\ref{fig.isBlocked} and used in rule 
\textbf{and-adv2}, requiring the left branch $p$ to be blocked in order to 
allow the right transition from $q$ to $q'$.
%
An expression becomes blocked when all of its trails in parallel hang in 
$awaiting$ and $emitting$ expressions.

\begin{figure}[t]
{\small
\begin{align*}
  isBlocked(n,a:S, awaiting(b,m)) &= (a \neq b \1\vee\1 m = n)   \\
  isBlocked(n,S, emitting(s))    &= (|S| \neq s)                     \\
  isBlocked(n,S, (p~;~q))        &= isBlocked(n,S,p)             \\
  isBlocked(n,S, (p~@~loop~q))   &= isBlocked(n,S,p)             \\
  isBlocked(n,S, (p~and~q))      &= isBlocked(n,S,p) \wedge
                                    isBlocked(n,S,q)             \\
  isBlocked(n,S, (p~or~q))       &= isBlocked(n,S,p) \wedge
                                    isBlocked(n,S,q)             \\
  isBlocked(n,S, \_)             &= false \2  (mem,await,      \\
                                  &    \5\5\5\2 emit,break,if,loop)   %\\
\end{align*}
}%
\rule{14cm}{0.37pt}
\caption{
The recursive predicate $isBlocked$ is true only if all branches in parallel 
are hanged in $awaiting$ or $emitting$ expressions that cannot transit.
\label{fig.isBlocked}
}
\end{figure}

\begin{figure}[t]
{\small
\begin{align*}
  clear( fin~p )       &= p                   \\
  clear( p~;~q )       &= clear(p)            \\
  clear( p~@~loop~q) ) &= clear(p)            \\
  clear( p~and~q )     &= clear(p)~;~clear(q) \\
  clear( p~or~q )      &= clear(p)~;~clear(q) \\
  clear( \_ )          &= mem(id)
\end{align*}
}%
\rule{14cm}{0.37pt}
\caption{
The function $clear$ extracts $fin$ expressions in parallel and put their 
bodies in sequence.
\label{fig.formal.clear}
}
\end{figure}

The last two rules \textbf{and-brk1} and \textbf{and-brk2} deal with a $break$ 
in each of the sides in parallel.
A $break$ should terminate the whole composition in order to escape the 
innermost loop (\emph{aborting} the other side).
%
The $clear$ function in the rules, defined in Figure~\ref{fig.formal.clear}, 
concatenates all active $fin$ bodies of the side being aborted (to execute 
before the $and$ rejoins).
Note that there are no transition rules for $fin$ expressions.
This is because once reached, an $fin$ expression only executes when it is 
aborted by a trail in parallel.
In Section~\ref{sec.formal.fins}, we show how an $fin$ is mapped to a 
finalization block in the concrete language.
%
Note that there is a syntactic restriction that an $fin$ body cannot $emit$ or 
$await$---they are guaranteed to completely execute within a reaction chain.

Most rules for parallel $or$ compositions are similar to $and$ compositions:

{ \setlength{\jot}{20pt}
\begin{eqnarray*}
& \frac
    {\DS \LL S,p \RR \ST \LL S',p' \RR }
%   -----------------------------------------------------------
    {\DS \LL S, (p~or~q) \RR \ST \LL S', (p'~or~q) \RR }
    & \textbf{(or-adv1)}   \\
%%%
& \frac
    {\DS isBlocked(n,S,p) \1,\2 \LL S,q \RR \ST \LL S',q' \RR }
%   -----------------------------------------------------------
    {\DS \LL S (p~or~q) \RR \ST \LL S', (p~or~q') \RR }
    & \textbf{(or-adv2)}   \\
%%%
& \LL S, (mem(id)~or~q) \RR \ST \LL S, clear(q) \RR
    & \textbf{(or-nop1)}   \\
%%%
& \frac
    {\DS isBlocked(n,S,p) }
%   -----------------------------------------------------------
    {\DS \LL S, (p~or~mem(id)) \RR \ST \LL S, clear(p) \RR }
    & \textbf{(or-nop2)}   \\
%%%
& \LL S, (break~or~q) \ST \LL S, (clear(q)~;~break) \RR
    & \textbf{(or-brk1)}   \\
%%%
& \frac
    {\DS isBlocked(n,S,p) }
%   -----------------------------------------------------------
    {\DS \LL S, (p~or~break) \RR \ST \LL S, (clear(p)~;~break) \RR }
    & \textbf{(or-brk2)}   %\\
\end{eqnarray*}
}

For a parallel $or$, the rules \textbf{or-nop1} and \textbf{or-nop2} must 
terminate the composition, and also apply the function $clear$ to the aborted 
side, in order to properly finalize it.

A reaction chain eventually blocks in $awaiting$ and $emitting$ expressions in 
parallel trails.
%
If all trails hangs only in $awaiting$ expressions, it means that the program 
cannot advance in the current reaction chain.
%
However, $emitting$ expressions should resume their continuations of previous 
$emit$ in the ongoing reaction, they are just hanged in lower stack indexes 
(see rule \textbf{emit}).
%
Therefore, we define another relation that behaves as the previous if the 
program is not blocked, and, otherwise, pops the stack:
%
$$
\frac
    { \DS \LL S,p \RR \ST                   \LL S',p' \RR }
%   -----------------------------------------------------------
    {     \LL S,p \RR \xRightarrow[~~n~~]{} \LL S',p' \RR }
%
\5\5\5
%
\frac
    { \DS isBlocked(n,\1s:S,\1p) }
%   -----------------------------------------------------------
    { \LL s:S,p \RR \xRightarrow[~~n~~]{} \LL S,p \RR }
$$
%
To describe a \emph{reaction chain} in \CEU, i.e., how a program behaves in 
reaction to a single external event, we use the reflexive transitive closure of 
this relation:
%
$$
    \LL S,p \RR \xRightarrow[~~n~~]{*} \LL S',p' \RR
$$
%
Finally, to describe the complete execution of a program, we need multiple 
``invocations'' of reaction chains, incrementing the sequence number:
%
\begin{align*}
\LL [e1], p \RR
    & \xRightarrow[~~1~~]{*}
\LL [  ], p' \RR
\\
\LL [e2], p' \RR
    & \xRightarrow[~~2~~]{*}
\LL [  ], p'' \RR
\\
& ...
\end{align*}
%
Each invocation starts with an external event at the top of the stack and 
finishes with a modified program and an empty stack.
After each invocation, the sequence number is incremented.

\section{Concrete language mapping}

Although the reduced syntax presented in Figure~\ref{lst.formal.syntax} is 
similar to the concrete language in Figure~\ref{lst.syntax}, there are some 
significant mismatches between \CEU and the formal semantics that require some 
clarification.
In this section, we describe an informal mapping between the two.

Most statements from \CEU map directly to the formal semantics, e.g.,
$
    \code{if}      \mapsto if   ,\2
    \code{';'}     \mapsto ~';' ,\2
    \code{loop}    \mapsto loop ,\2
    \code{par/and} \mapsto and  ,\2
    \code{par/or}  \mapsto or
$.
(Again, we are not considering side-effects, which are all mapped to the $mem$ 
semantic construct.)

\subsection{await and emit}

The \code{await} and \code{emit} primitives of \CEU are slightly more complex 
in comparison to the formal semantics, as they support communication of values 
between emits and awaits.
In the two-step translation below, we start with the program in \CEU, which 
communicates the value $1$ between the \code{emit} and \code{await} in parallel 
(left-most code).
In the intermediate translation, we include the shared variable \code{e\_} to 
hold the value being communicated between the two trails in order to simplify 
the \code{emit}.
Finally, we convert the program into the equivalent in the formal semantics, 
translating side-effect statements into $mem$ expressions:

\begin{figure}[h!]
\begin{minipage}[t]{0.32\linewidth}
\begin{lstlisting}
par/or do
  <...>
  emit e => 1;
with
  v = await e;
  _printf("%d\n",v);
end
\end{lstlisting}
\end{minipage}
%
\begin{minipage}[t]{0.32\linewidth}
\begin{lstlisting}
par/or do
  <...>
  e_ = 1;
  emit e;
with
  await e;
  v = e_;
  _printf("%d\n",v);
end
\end{lstlisting}
\end{minipage}
%
\begin{minipage}[t]{0.34\linewidth}
\begin{lstlisting}
<...> ; mem ; emit(e)
        or
await(e) ; mem ; mem
\end{lstlisting}
\end{minipage}
\end{figure}

Note that a similar translation is required for external events, i.e., each 
external event has a corresponding variable that is explicitly set by the 
environment before each reaction chain.

\subsection{First-class timers}

To encompass first-class timers, we need a special \code{TICK} event that 
should be intercalated with each other event occurrence in an application:

\begin{align*}
\LL [TICK], p \RR
    & \xRightarrow[~~1~~]{*}
\LL [    ], p' \RR
\\
\LL [e1], p' \RR
    & \xRightarrow[~~2~~]{*}
\LL [  ], p'' \RR
\\
\LL [TICK], p'' \RR
    & \xRightarrow[~~3~~]{*}
\LL [    ], p''' \RR
\\
\LL [e2], p''' \RR
    & \xRightarrow[~~4~~]{*}
\LL [  ], p'''' \RR
\\
& ...
\end{align*}

The \code{TICK} event has an associated variable \code{TICK\_} (as illustrated 
in the previous section) with the time elapsed between the two occurrences of 
external events.

The translation in two steps from a timer await to the semantics is as follows:

\noindent
\begin{minipage}[t]{0.30\linewidth}
\begin{lstlisting}
dt = await 10ms;
\end{lstlisting}
\end{minipage}
%
\begin{minipage}[t]{0.37\linewidth}
\begin{lstlisting}
var int tot = 10000;
loop do
    await TICK;
    tot = tot - TICK_;
    if tot <= 0 then
        dt = tot;
        break;
    end
end
\end{lstlisting}
\end{minipage}
%
\begin{minipage}[t]{0.30\linewidth}
\begin{lstlisting}
mem;
loop(
    await(TICK);
    mem;
    if mem then
        mem;
        break
    else
        nop
)
\end{lstlisting}
\end{minipage}

\subsection{Finalization blocks}
\label{sec.formal.fins}

The biggest mismatch between \CEU and the formal semantics is regarding 
finalization blocks, which require more complex modifications in the program
for a proper mapping using the $fin$ semantic construct.
The code that follows uses a \code{finalize} to safely \code{\_release} the 
reference to \code{ptr} kept after the call to \code{\_hold}:

\begin{lstlisting}
do
    var int* ptr = <...>;
    await A;
    finalize
        _hold(ptr);
    with
        _release(ptr);
    end
    await B;
end
\end{lstlisting}

In the translation to the semantics, the first required modification is to 
catch the \code{do-end} termination to run the finalization code.
For this, we translate the block into a \code{par/or} with the original body in 
parallel with a $fin$ to run the finalization code:

\begin{lstlisting}
par/or do
    var int* ptr = <...>;
    await A;
    _hold(ptr);
    await B;
with
    { fin
        _release(ptr); }
end
\end{lstlisting}

In this intermediate code (mixing the syntaxes), the $fin$ body will execute
whenever the \code{par/or} terminates, either normally (after the \code{await 
B}) or aborted from an outer composition (rules \textbf{and-brk1}, 
\textbf{and-brk2}, \textbf{or-nop1}, \textbf{or-nop2}, \textbf{or-brk1}, and 
\textbf{or-brk2} in the semantics).
%
%Note the choice for a \code{par/or} which terminates only with the first 
%trail.
%
However, the $fin$ will also (incorrectly) execute even if the call to 
\code{\_hold} is not reached in the body due to an abort before awaking from 
the \code{await A}.
%It may happen that the call to \code{\_release} occurs without a previous call 
%to \code{\_hold}.
%
To deal with this issue, for each $fin$ we need a corresponding flag to keep 
track of code that needs to be finalized:

\begin{lstlisting}[numbers=left,xleftmargin=2em]
f_ = 0;
par/or do
    var int* ptr = <...>;
    await A;
    _hold(ptr);
    f_ = 1;
    await B;
with
    { fin
        if f_ then
            _release(ptr);
        end }
end
\end{lstlisting}

The flag is initially set to false (line 1), avoiding the finalization code to 
execute (lines 9-12).
Only after the call to \code{\_hold} (line 5) that we set the flag to true 
(line 6) and enable the $fin$ body to execute.
%
The complete translation from the original example in \CEU is as follows:

\begin{lstlisting}
mem;    // f_ = 0
(
   mem;         // ptr = <...>
   await(A);
   mem;         // _hold(ptr)
   mem;         // f_ = 1
   await(B);
or
   fin
     if mem then    // if f_
        mem         // release _ptr
     else
        nop
)
\end{lstlisting}

\begin{comment}

There are also some restriction on valid programs in \CEU, which the semantic 
allows:
- loops
- awaits in fins
- local scopes

handled in separate, in a parsing phase that is not related to the execution

%%%%%%%%%%%%%%%%%%%%%%%%%%%%%%

The semantic rules are continuously applied (note the `$*$' klenee operator) 
until
consider only the top of the stack and continuously apply transformations to 
the program until it blocks and no rules can be applied.
The \emph{isBlocked} predicate of Figure~\ref{fig.isBlocked} identifies a 
blocked program based on its structure and event at the top of the stack.
xxx
A program becomes blocked when all parallel branches are hanged in $awaiting$, 
$stacked$, and/or $emitting$ primitives, as defined in 
Figure~\ref{fig:isBlocked}.
an \code{awaiting} is unblocked only if its event matches the top of the stack
the \code{emitting} primitive only proceeds once its stack level is restored.

\begin{itemize}
\item The program is awaiting in all trails, i.e., function $pop$ returns 
$(0,\{\})$.
\item The program terminates, i.e., the small-step rules transform the whole 
program into a $mem(id)$.
\end{itemize}
%
In Section~\ref{XXX} we show that by imposing syntactic restrictions to 
programs, reaction chains always reach one of these conditions in a finite 
number of steps, meaning that reactions to the environment always execute in 
bounded time.

To be compliant with the reactive nature of \CEU, we assume that all programs 
start awaiting the main event ``$\$$'', which is emitted once by the 
environment on startup, i.e., $(i,E)=(1,\{\$\})$ for the very first big step.

As briefly introduced, small-step rules continuously apply transformations to 
unblocked trails.
A program becomes blocked when all parallel branches are hanged in $awaiting$, 
$stacked$, and/or $emitting$ primitives, as defined in 
Figure~\ref{fig:isBlocked}.

All small-step rules are associated with the current (deepest) stack depth 
level $i$ acquired from the previous big step.

%%%%%%%%%%%%%%%%%%%%%%%%%%%%%%%%%%%%%%%%%%%%%%%%%%%%%%%%%%%%%%%%%%%%%%%%%%%%%%%

\end{comment}
