\begin{comment}
System-level development for WSNs basically consists of abstracting access to 
the hardware and designing specially tweaked network protocols to be further 
integrated as services in higher-level applications or macro-programming 
systems~\cite{wsn.state_of_art,wsn.tos,wsn.survey}.
%The external environment plays an important, as it interacts with applications 
%permanently (and concurrently) by issuing events that represent expiring 
%timers, messages arrivals, sensor readings, etc.

Our design is first compromised with the main principles that govern WSN 
development: \emph{resource minimization} and \emph{bug prevention}, as defined 
by Levis~\cite{wsn.decade}.

\end{comment}

Wireless Sensor Networks (WSNs) are composed of a large number of tiny devices 
(known as ``motes'') capable of sensing the environment and communicating.
They are usually employed to continuously monitor physical phenomena in large 
or unreachable areas, such as wildfire in forests and air temperature in 
buildings.
Each mote features limited processing capabilities, a short-range radio link, 
and one or more sensors (e.g. light and temperature) \cite{wsn.survey}.

WSNs are usually designed with safety and (soft) real-time requirements under 
constrained hardware platforms.
At the same time, developers demand effective programming abstractions, ideally 
with unrestricted access to low-level functionality.
%
These particularities impose a challenge to WSN-language designers, who must 
provide a comprehensive set of features requiring correct and predicable 
behavior under platforms with limited memory and CPU.
As a consequence, WSN languages either lack functionality or fail to offer a 
small and reliable programming environment.

System-level development for WSNs commonly follows one of three major 
programming models: \emph{event-driven}, \emph{multi-threaded}, or 
\emph{synchronous}.
%
In event-driven programming \cite{wsn.tos,wsn.contiki}, each external event can 
be associated with a short-lived function callback to handle a reaction to the 
environment.
This model is efficient,
% for the severe resource constraints of WSNs,
but is known to be difficult to 
program~\cite{sync_async.cooperative,wsn.protothreads}.
%
Multi-threaded systems emerged as an alternative in WSNs, providing traditional 
structured programming in multiple lines of 
execution~\cite{wsn.protothreads,wsn.mantisos}.
However, the development process still requires manual synchronization and 
bookkeeping of threads~\cite{sync_async.threadsproblems}.
%(e.g. create, start, and rejoin)
%
Synchronous languages~\cite{rp.twelve} have also been adapted to WSNs and offer 
higher-level compositions of activities with a step-by-step execution, 
considerably reducing programming efforts~\cite{wsn.sol,wsn.osm}.

Despite the increase in development productivity, WSN system languages still 
fail to ensure static safety properties for concurrent programs.
%
However, given the difficulty in debugging WSN applications, it is paramount to 
push as many safety guarantees to compile time as possible~\cite{wsn.decade}.
%
Shared-memory concurrency is an example of a widely adopted mechanism that 
typically relies on runtime safety primitives only.
For instance, current WSN languages ensure atomic memory access either through 
runtime barriers, such as mutexes and 
locks~\cite{wsn.mantisos,wsn.tinythreads}, or by adopting cooperative 
scheduling which also requires explicit yield points in the 
code~\cite{wsn.sol,wsn.protothreads}.
%Requiring explicit synchronization primitives lead to potential safety 
%hazards~\cite{sync_async.threadsproblems}.
%Enforcing cooperative scheduling is no less questionable, as it assumes that 
%\emph{all} accesses are dangerous.
In either case, there is no additional static guarantees or warnings about 
unsafe memory accesses.
%The bottom line is that existing languages cannot detect and enforce atomicity 
%only when they are required.

We believe that programming WSNs can benefit from a new language that takes 
concurrency safety as a primary goal, while preserving typical multi-threading 
features that programmers are familiarized with, such as shared memory 
concurrency.
%
We present \CEU%
\footnote{C\'eu is the Portuguese word for \emph{sky}.},
a synchronous system-level programming language that provides a reliable yet 
powerful set of abstractions for the development of WSN applications.
%
\CEU is based on a small set of control primitives similar to 
Esterel's~\cite{esterel.ieee91}, leading to implementations that more closely 
reflect program specifications.
%
As a main contribution, we propose a static analysis that permeates all 
language mechanisms and detects safety threats, such as concurrent accesses to 
shared memory and concurrent termination of threads, at compile time.
%
In addition, we introduce the following new safety mechanisms:
\emph{first-class timers} to ensure that timers in parallel remain synchronized 
(not depending on internal reaction timings);
\emph{finalization blocks} for local pointers going out of scope;
and \emph{stack-based communication} that avoids cyclic dependencies.
%
Our work focuses on \emph{concurrency safety}, rather than \emph{type safety}.%
\footnote{
We consider both safety aspects to be complimentary and orthogonal, i.e., 
type-safety techniques~\cite{wsn.safety} could also be applied to \CEU.
}

In order to enable the static analysis, programs in \CEU must suffer some 
limitations.
Computations that run in unbounded time (e.g., compression, image processing) 
cannot be elegantly implemented~\cite{rp.hypothesis}, and dynamic loading is 
forbidden.
%do not fit the zero-delay reaction hypothesis~\cite{rp.hypothesis}, and
%
%Also, dynamic support in the language, such as memory allocation and dynamic 
%loading, is forbidden.
%
However, we show that \CEU is sufficiently expressive for the context of WSN 
applications.
We successfully implemented the \emph{CC2420} radio driver~\cite{wsn.teps}, and 
the \emph{DRIP}, \emph{SRP}, and \emph{CTP} network protocols~\cite{wsn.teps}%
\footnote{
\emph{DRIP} is a data dissemination protocol to reliably deliver data to every 
node in the network.
\emph{SRP} is a routing protocol to deliver packets from an origin node to a 
destination node.
\emph{CTP} is a collectino protocol to deliver packets from any node to a 
collection roots in a network.
}
in \CEU.
In comparison to \emph{nesC}~\cite{wsn.nesc}, the implementations reduced the 
number of source code tokens by 25\%, with an increase in ROM and RAM below 
10\%.
%The radio driver also shows a responsiveness equivalent to \emph{nesC} in most 
%scenarios.

% TTT, mais coisas

The rest of the thesis is organized as follows:
%
Chapter~\ref{sec.models} introduces \CEU through comparisons with 
state-of-the-art languages representing the prevailing concurrency models used 
in WSNs.
%
Chapter~\ref{sec.ceu} details the design of \CEU, motivating and discussing the
safety aspects of each relevant language feature.
%
Chapter~\ref{sec.demos} presents two demo applications, exploring the safe and 
high-level programming style promoted by the language.
%
Chapter~\ref{sec.eval} evaluates the implementation of some network protocols 
in \CEU and compares some of its aspects with \emph{nesC} (e.g. memory usage 
and tokens count).
We also evaluate the responsiveness of the radio driver written in \CEU.
%
Chapter~\ref{sec.formal} presents a formal semantics of \CEU restricted to its 
control primitives, which comprises most novelties and challenging parts of our 
design.
%
Chapter~\ref{sec.impl} discusses key aspects of the implementation of \CEU, 
such as the static analysis algorithm and stackless lines of execution.
%
Chapter~\ref{sec.related} discusses related work to \CEU.
% TTT: future
%
Chapter~\ref{sec.conclusion} concludes the thesis and makes final remarks.
