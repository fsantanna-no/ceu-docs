\begin{comment}

\documentclass[11pt,a4paper]{article}
\usepackage[body={6.0in, 8.2in},left=1.25in,right=1.25in]{geometry}

\usepackage{fancyvrb}
\usepackage{verbatim}

\usepackage[pdftex]{graphicx}
\usepackage{subfigure}

\usepackage{listings}
\lstset{frame=tb}
\lstset{basicstyle=\scriptsize}
\lstset{tabsize=4}

\usepackage{amssymb}
\usepackage{amsmath}
\usepackage{amsfonts}
\usepackage{amsthm}

\newcommand{\1}{\;}
\newcommand{\2}{\;\;}
\newcommand{\3}{\;\;\;}
\newcommand{\5}{\;\;\;\;\;}
\newcommand{\ten}{\5\5}
\newcommand{\twenty}{\ten\ten}

\newcommand{\CEU}{\textsc{C\'{e}u}}
\newcommand{\code}[1] {{\small{\texttt{#1}}}}

\newcommand{\ST}{\xrightarrow[i]{}}
\newcommand{\BT}{\xRightarrow[(i,E)]{}}

\begin{document}

\section{Formalization of \CEU}

\end{comment}

\subsection{Abstract syntax}
\label{sec.sem.syntax}

Figure~\ref{lst:ceu:syn} shows the syntax for a subset of \CEU that is 
sufficient to describe all semantic peculiarities of the language.

\begin{figure}[t]
%\rule{8.5cm}{0.37pt}
{\small
\begin{verbatim}
  // primary primitives         // description
  nop(v)                        (constant value)
  mem                           (any memory access)
  await(e)                      (await event `e')
  emit(e)                       (emit event `e')
  break                         (loop escape)

  // compound statements
  if mem then p else q          (conditional)
  p ; q                         (sequence)
  loop p                        (repetition)
  p and q                       (par/and)
  p or q                        (par/or)

  // derived by semantic rules
  awaiting(e)                   (previously awaiting `e')
  stacked(i)                    (stack depth mark)
  emitting(i,e)                 (emitting `e')
  p @ loop p                    (unwinded loop)
\end{verbatim}
}%
\caption{ Simplified syntax of \CEU.
\label{lst:ceu:syn}
}
\end{figure}

A $nop$ represents a terminated computation associated with a constant value.
The $mem$ primitive represents any memory accesses, assignments, and $C$ 
function calls.
As the challenging parts of \CEU reside on its control structures, we are not 
concerned here with a precise semantics for side effects, but only with their 
occurrences in programs.
We refer back to side effects when discussing determinism in 
Section~\ref{sec.safety.det}.

The $await$ and $emit$ primitives are responsible for the reactive nature of 
\CEU.
An $await$ can refer either to an external or internal event, while an $emit$ 
can only refer to an internal event.

The semantic rules to be presented generate three statements that the 
programmer cannot write:
the primitives $awaiting$, $stacked$, and $emitting$ avoid the immediate 
matching of emits and awaits and are used as an artifice to provide the desired 
stacked behavior for internal events.
A $loop$ is expanded with the special \code{`@'} separator (instead of 
\code{`;'}) to properly bind $break$ statements inside $p$ to the enclosing 
loop.
\subsection{Operational semantics}
\label{sec.sem}

In the remaining of this section, we present an operational semantics to 
formally describe a \emph{reaction chain} in \CEU, i.e., how a program behaves 
in reaction to a single external event.%
\footnote{We could extend the semantics to describe the full execution of a 
program by holding new incoming external events in a queue and processing them 
in consecutive reaction chains that never overlap.}

The semantics is split in two sets of rules: \emph{big-step} and 
\emph{small-step} rules.
First, we apply a single big step to awake all trails awaiting the broadcast 
external event.
Then, we continuously apply small steps until all trails await and/or emit.
The next big step awakes all previously awaiting trails matching the emits at 
once.
The two set of rules are interleaved to achieve a complete reaction chain, 
terminating when the program either terminates or awaits in all trails.

In order to provide the desired stacked execution for internal events, the 
semantic rules are associated with an index $i$ that represents the current 
runtime stack depth level.
An emit creates a deeper level $i+1$ and is deferred to be matched in the next 
big step.
The emit continuation (i.e., the statement that follows the emit) remains at 
stack level $i$, as it can only execute after the program completely reacts to 
the event.
Awakened trails may emit new events that will increase the stack depth ($i+2$ 
and so on);
hence, only after the stack unrolls to depth $i$ that the original emitting 
trail continues.

A complete reaction chain to an external event is formalized as follows:
%
$$
%p \xrightarrow[big]{(i,E) = (0,\{ext\})}  \3
(\3
p                                        \1
    \xRightarrow[~(i,E)=pop(p)~]{}     \1
p'
    \xrightarrow[~~i~~]{~~*~~}           \1
p''
\3)*
$$

A big step is represented with the double arrow, where the tuple $(i,E)$ is the 
set $E$ of emitting events to be matched at deepest depth $i$.

For the initial big step, function $pop$ returns the single external event 
emitted at starting depth $1$.
For further big steps, $pop$ is defined in Figure~\ref{fig:pop} and returns the 
deepest stack level and all emitted internal events (if any) from the previous 
small-step sequence.
%
%\footnote{$pop$ needs only to be defined for blocked primitives ($awaiting$ 
%and $stacked$), as small-step sequences that precede big steps always reach 
%these primitives.}
% TODO: colocar na definicao

\begin{figure}[t]
{\small
\begin{align*}
  pop(awaiting(e))   &= (-1, \{\})              \\
  pop(stacked(i))    &= ( i, \{\})              \\
  pop(emitting(i,e)) &= ( i, \{e\})             \\
  pop(p~;~q)         &= pop(p)                  \\
  pop(p~@~loop~q)    &= pop(p)                  \\
  pop(p~and/or~q)    &= (max(j,k), F\cup{}G)    \\
                     & \ten where~(j,F)=pop(p)  \\
                     & \ten\2\2 and~(k,G)=pop(q)
\end{align*}
\emph{(*) Other primitives (nop,mem,await,emit,break,if,loop) do not apply.}
}%
\caption{ The recursive definition for $pop$.
\label{fig:pop}
}
\end{figure}

A small-step sequence is represented with the single arrow and is associated 
with the same depth $i$ from the previous big step, which identifies the 
current (deepest) stack depth.

The sets of rules are interleaved until one of the two possible terminating 
conditions for a reaction chain apply:

\begin{itemize}
\item The program is awaiting in all trails, i.e., function $pop$ returns 
$(-1,\{\})$.
\item The program terminates, i.e., the small-step rules transform the whole 
program into a $nop$.
\end{itemize}
%
In Section~\ref{sec.safety.bounded} we show that, by imposing syntactic 
restrictions to programs, reaction chains always reach one of these conditions 
in a finite number of steps, meaning that reactions to the environment always 
execute in bounded time.

To be compliant with the reactive nature of \CEU, we assume that all programs 
start awaiting the main event ``$\$$'', which is emitted once by the 
environment on startup, i.e., $(i,E)=(1,\{\$\})$ for the very first big step.

\subsubsection{Small-step rules}
\label{sec.sem.small}

As briefly introduced, small-step rules continuously apply transformations to 
unblocked trails.
A program becomes blocked when all parallel branches are hanged in $awaiting$, 
$stacked$, and/or $emitting$ primitives, as defined in 
Figure~\ref{fig:isBlocked}.

% TODO: [|

\begin{figure}[t]
{\small
\begin{align*}
  isBlocked(awaiting(e))   &= true                             \\
  isBlocked(stacked(i))    &= true                             \\
  isBlocked(emitting(i,e)) &= true                             \\
  isBlocked(p~;~q)         &= isBlocked(p)                     \\
  isBlocked(p~@~loop~q)    &= isBlocked(p)                     \\
  isBlocked(p~and~q)       &= isBlocked(p) \wedge isBlocked(q) \\
  isBlocked(p~or~q)        &= isBlocked(p) \wedge isBlocked(q) \\
  isBlocked(*)             &= false \2  (nop,mem,await,        \\
                           &    \5\5\5\2 emit,break,if,loop)   %\\
\end{align*}
}%
\caption{ The recursive predicate $isBlocked$.
\label{fig:isBlocked}
}
\end{figure}

All small-step rules are associated with the current (deepest) stack depth 
level $i$ acquired from the previous big step.

We start with the small-step rules for the primary primitives:
%
\begin{eqnarray*}
& mem \ST nop(v),~~(v~is~nondet)
    & \textbf{(mem)}        \\
%%%
& await(e) \ST stacked(0)~;~awaiting(e)
    & \textbf{(await)}      \\
%%%
& emit(e) \ST emitting(i+1,e)~;~stacked(i)
    & \textbf{(emit)}       %\\
\end{eqnarray*}

A $mem$ operation terminates with a nondeterministic value (e.g., the value of 
a variable or a $C$ function call).

An $await$ is stacked with the lowest possible depth, being transformed in an 
$awaiting$ only in the end of the reaction chain.
This way, a trail that reaches an $await$ can only react to an $emit$ that 
occurs in further reaction chains.

An $emit$ is deferred to a deeper depth to be applied in the next big step.
The rule actually transforms an $emit$ in two primitives in sequence:
the $emitting(i+1,e)$ defers the immediate matching, while $stacked(i)$ holds 
the trail in the current depth, resuming only after the complete reaction to 
the event.

Given that small-step rules execute at the current deepest level and that rule 
\textbf{emit} is the only one that increases the stack depth, the next big step 
will necessarily take all deferred emits.
This explains why the definition for $pop$ in Figure~\ref{fig:pop} blindly 
takes the union of all emitted events without considering their depths.

All other primitives ($nop$, $break$, $awaiting$, $stacked$, and $emitting$) 
represent terminated or blocked trails and, therefore, have no associated 
small-step rules.

The rules for conditionals and sequences are straightforward:
%
\begin{eqnarray*}
& \frac
    { m \ST m' }
% -----------------------------------------------------------
    { (if~m~then~p~else~q) \ST (if~m'~then~p~else~q) }
    & \textbf{(if-adv)}       \\
%%%
& (if~nop(v)~then~p~else~q) \ST p \1,\3 (v \neq 0)
    & \textbf{(if-true)}       \\
%%%
& (if~nop(0)~then~p~else~q) \ST q
    & \textbf{(if-false)}       %\\
%%%
%& (if~mem~then~p) \ST (if~mem~then~p~else~nop)
    %& \textbf{(if-else)}
\end{eqnarray*}
%%%
\begin{eqnarray*}
& \frac
    { p \ST p' }
%   -----------------------------------------------------------
    { (p~;~q) \ST (p'~;~q) }
    & \textbf{(seq-adv)}      \\
%%%
& (nop~;~q) \ST q
    & \textbf{(seq-cst)}      \\
%%%
& (break~;~q) \ST break
    & \textbf{(seq-brk)}     %\\
\end{eqnarray*}
%
%A conditional proceeds to either $p$ or $q$, depending on the evaluation of 
%$mem$ (rules \textbf{if-true} and \textbf{if-false}).
%Note that an empty $else$ branch is substituted by a $nop$ (rule 
%\textbf{if-else}).
%
%Rule \textbf{seq-1} advances composition $p$ until it becomes either a $nop$ 
%(proceeding to $q$ in rule \textbf{seq-2}), or a $break$ (ignoring $q$ in rule 
%\textbf{seq-3}).

Given that the semantics focus on control, note that rules \textbf{if-true} and 
\textbf{if-false} are the only to query $nop$ values.
For all other rules, we omit these values (e.g., \textbf{seq-cst}).

The rules for loops are analogous to sequences, but use \code{`@'} as 
separators to properly bind breaks to their enclosing loops:
%
\begin{eqnarray*}
& { (loop~p) \ST (p~@~loop~p) }
    & \textbf{(loop-expd)}       \\
%%%
& \frac
    { p \ST p' }
% -----------------------------------------------------------
    { (p~@~loop~q) \ST (p'~@~loop~q) }
    & \textbf{(loop-adv)}    \\
%%%
& (nop~@~loop~p) \ST loop~p
    & \textbf{(loop-cst)}    \\
%%%
& (break~@~loop~p) \ST nop
    & \textbf{(loop-brk)}
\end{eqnarray*}

When a program first encounters a $loop$, it first expands its body in sequence 
with itself (rule \textbf{loop-expd}).
Rules \textbf{loop-adv} and \textbf{loop-cst} are similar to rules 
\textbf{seq-adv} and \textbf{seq-cst}, advancing the loop until it reaches a 
$nop$.
However, what follows the loop is the loop itself (rule \textbf{loop-cst}).
Rule \textbf{loop-brk} escapes the enclosing loop, transforming everything into 
a $nop$.
Note that if we used \code{`;'} as a separator in loops, rules 
\textbf{loop-brk} and \textbf{seq-brk} would conflict.

The small-step rules for parallel compositions advance trails independently and 
require that both sides either block or terminate.
% to advance to the next big step.
For an $and$, if one of the sides terminate, the composition is simply
substituted by the other side.
For an $or$, if one of the sides terminate, the other side must advance until 
it blocks.
Then, the whole composition terminates and the blocked side is \emph{killed}, 
i.e., all its $awaiting$, $stacked$, and $emitting$ primitives are eliminated 
with the rule transformation.

A similar situation occurs when an $and$ or $or$ reach a $break$ in either of 
the sides.
In this case, the enclosing $loop$ must terminate and both sides are killed, 
transforming the whole composition into a $break$.
As the rules make explicit, a trail can only be killed after it terminates or 
blocks.

Follow the rules for a parallel $and$:
%
\begin{eqnarray*}
& \frac
    { p \ST p' }
%   -----------------------------------------------------------
    { (p~and~q) \ST (p'~and~q) }
    & \textbf{(and-adv1)}      \\
%%%
& \frac
    { q \ST q' }
%   -----------------------------------------------------------
    { (p~and~q) \ST (p~and~q') }
    & \textbf{(and-adv2)}      \\
%\end{eqnarray*}
%%%
%\begin{eqnarray*}
& (nop~and~q) \ST q
    & \textbf{(and-cst1)}   \\
%%%
& (p~and~nop) \ST p
    & \textbf{(and-cst2)}   \\
%%%
& \frac
    { q=break \1\vee\1 isBlocked(q) }
%   -----------------------------------------------------------
    { (break~and~q) \ST break }
    & \textbf{(and-brk1)}   \\
%%%
& \frac
    { p=break \1\vee\1 isBlocked(p) }
%   -----------------------------------------------------------
    { (p~and~break) \ST break }
    & \textbf{(and-brk2)}   %\\
\end{eqnarray*}

Rules \textbf{and-cst1} and \textbf{and-cst2} handle the termination of one of 
the sides, substituting the whole composition by the other side.
Rules \textbf{and-brk1} and \textbf{and-brk2} handle the special case for 
reaching a $break$, in which the other blocked side is killed by transforming 
the whole composition becomes into a $break$ to terminate the enclosing loop.
For instance, the program \code{loop(break~and~emit(a))} never emits $a$, 
because the $emit$ blocks (rule \textbf{emit}) and is then killed (rule 
\textbf{and-brk1}).

The rules for a parallel $or$ are slightly different:
%
\begin{eqnarray*}
& \frac
    { p \ST p' }
%   -----------------------------------------------------------
    { (p~or~q) \ST (p'~or~q) }
    & \textbf{(or-adv1)}   \\
%%%
& \frac
    { q \ST q' }
%   -----------------------------------------------------------
    { (p~or~q) \ST (p~or~q') }
    & \textbf{(or-adv2)}   \\
%%%
& \frac
    { q=nop \1\vee\1 isBlocked(q) }
%   -----------------------------------------------------------
    { (nop~or~q) \ST nop }
    & \textbf{(or-cst1)}   \\
%%%
& \frac
    { p=nop \1\vee\1 isBlocked(p) }
%   -----------------------------------------------------------
    { (p~or~nop) \ST nop }
    & \textbf{(or-cst2)}   \\
%%%
& \frac
    { q=nop \1\vee\1 q=break \1\vee\1 isBlocked(q) }
%   -----------------------------------------------------------
    { (break~or~q) \ST break }
    & \textbf{(or-brk1)}   \\
%%%
& \frac
    { p=nop \1\vee\1 p=break \1\vee\1 isBlocked(p) }
%   -----------------------------------------------------------
    { (p~or~break) \ST break }
    & \textbf{(or-brk2)}   %\\
\end{eqnarray*}

Rules \textbf{or-cst1} and \textbf{or-cst2} terminate the $or$ when at least 
one of the sides is a $nop$.
For instance, the program \code{loop(nop~and~emit(a))} never emits $a$, because 
the $emit$ blocks (rule \textbf{emit}) and is then killed (rule 
\textbf{or-cst1}).
%Note that these rules enforce both sides to advance until they either 
%terminate or block.
%, what necessarily happens because all unblocked primitives have associated 
%small-step rules.
Rules \textbf{or-brk1} and \textbf{or-brk2} are similar to their $and$ 
counterparts, with the additional remark that a $break$ has preference over a 
$nop$, i.e., when they appear on each side, the whole composition becomes a 
$break$ instead of a $nop$.

%(TODO: orthogonal preemption)

Note that rule \textbf{mem} and the pairs \textbf{and-adv1}/\textbf{and-adv2} 
and \textbf{or-adv1}/\textbf{or-adv2} bring nondeterminism to the semantics of 
\CEU.
%Also, rule \textbf{loop-exp} may expand the program indefinitely, leading to 
%unbounded execution for \CEU programs.
However, in Section~\ref{sec.safety.det} we discuss how to detect programs with 
deterministic behavior (even with nondeterminism in $mem$ operations and 
scheduling), refusing all other programs at compile time.

\subsubsection{Big-step rules}
\label{sec.sem.big}

The big-step semantics matches $emitting$ and $awaiting$ trails, providing 
broadcast communication in the language.
It is important to use a big-step operational semantics in order to apply 
transformations in parallel, all at once.
Emits with no matching awaits are simply discarded, characterizing the 
unbuffered communication typically adopted in synchronous languages.

Big-step rules are associated with a tuple $(i,E)$ that represents the set of 
occurring events $E$ triggered at stack depth $i$.

\begin{comment}
At the beginning of each reaction chain, only a single external event can be 
active, as required by \CEU's synchronous execution model.
Hence, initially $(i,E) = (0,\{ext\})$, where $ext$ is the external event 
triggered at starting depth $0$.

For further big steps following small steps, the tuple $(i,E)$ is acquired from 
the recursive function $pop$, which takes the deepest stack depth and groups 
all stacked emits (if any) together:

As explained for small-step rule \textbf{emit}, all stacked emits necessarily 
occur at deepest depth.
Note also that $pop$ does not need to be defined for unblocked primitives, as a 
small-step sequence that precedes a big step only terminates when the program 
becomes blocked.

We now present the big-step semantic rules.
\end{comment}

We start with the rules to check if deferred emits match previously awaiting 
trails:
%
\begin{eqnarray*}
& awaiting(e) \BT nop \1,\3 (e \in E)
    & \textbf{(Await-awk)}    \\
%%%
& awaiting(e) \BT awaiting(e) \1,\3 (e \notin E)
    & \textbf{(Await-rep)}    \\
\end{eqnarray*}

The $stacked$ and $emitting$ primitives are ``popped'' if they are at the 
deepest level $i$ returned by function $pop$:

\begin{eqnarray*}
& stacked~(i) \BT nop \5
    & \textbf{(Stack)}  \\
%
& emitting~(i,e) \BT nop \5
    & \textbf{(Emitting)}  %\\
\end{eqnarray*}

\begin{comment}
A deferred emit cannot unblock immediately because the just matched trails 
(rule \textbf{Await-awk}) must all react before the emit continuation proceeds.
Hence, rule \textbf{Stack} specifies that the emit continuation remains blocked 
at the same (low) depth.
This way, awaken trails will execute in the next small-step sequence and new 
emits will be deferred at deeper depths (note the \code{i+1} in small-step rule 
\textbf{emit}).
Eventually, no new emits will take place, and function $pop$ will pop the 
continuations with the highest depth, which will proceed through rule 
\textbf{Cont}, providing the desired stacked execution for internal events.
\end{comment}

To conclude the big-step semantics, the rules for compound statements advance 
their subparts all at once:
%
\begin{eqnarray*}
& \frac
    { p \BT p' }
%   -----------------------------------------------------------
    { (p~;~q) \BT (p'~;~q) }
    & \textbf{(Seq)}    \\
%%%
& \frac
    { p \BT p' }
%   -----------------------------------------------------------
    { (p~@~loop~q) \BT (p'~@~loop~q) }
    & \textbf{(Loop)}   \\
%%%
& \frac
    { p \BT p' \5 q \BT q' }
%   -----------------------------------------------------------
    { (p~and~q) \BT (p'~and~q') }
    & \textbf{(And)}    \\
%%%
& \frac
    { p \BT p' \5 q \BT q' }
%   -----------------------------------------------------------
    { (p~or~q) \BT (p'~or~q') }
    & \textbf{(Or)}     %\\
\end{eqnarray*}

Note that there are no rules for $mem$, $break$, $emit$, $nop$, and $if$ 
because none of these represent blocked primitives, and hence, never appear in 
a big step.

\begin{comment}

\textbf{(TODO) An example:}

{\small
\begin{verbatim}
    mem(1); emit a; mem(2); emit b; mem(3)
or (
    loop (mem(4); await a)
or
    await b; mem(5)
)
\end{verbatim}
}

$$
p = program
\ten
ext = external~event
$$

$$
*^1: \1 until \3 isBlocked(p'') \1\vee\1 (p''=nop)
$$
$$
*^2: \1 until \3 pop(p'')=(0,\emptyset)
$$

\subsection{Examples}

$$
\textbf{(seq)} \5
    \frac
    { \textbf{(await 1)} \3 await~A \xrightarrow{(A,0)} delay~0 }
%   -----------------------------------------------------------
    { (await~A~;~await~A) \xrightarrow{(A,0)} (delay~0~;~await~A) }
$$

$$
\textbf{(seq 1)} \5
    \frac
    {\textbf{(delay)} \5 (delay~0) \xrightarrow{0}
        nop \1\nearrow^{~\emptyset}}
%   -----------------------------------------------------------
    { (delay~0~;~await~A) \xrightarrow{0} (nop~;~await~A)
        \1\nearrow^{~\emptyset} }
$$

$$
\textbf{(seq 2)} \3 (nop~;~await~A) \xrightarrow{i}
                        await~A \1\nearrow^{~\emptyset}
$$

$$
\textbf{(loop 1)} \5
    \frac
    {\textbf{(if)} \5
        (if~...) \xrightarrow{(\_,0)} (if~...)}
%   -----------------------------------------------------------
    { loop~(if~...) \xrightarrow{(\_,0)}
        (if~mem(v)~then~break~else~await~A)~@~loop~(if~...)) }
$$

$$
\textbf{(if 0)} \3
    (if~mem(0)~then~p~else~q) \xrightarrow{i} q \1\nearrow^{~\emptyset}
\ten
\textbf{(if 1)} \3
    (if~mem(v)~then~p~else~q) \xrightarrow{i} p \1\nearrow^{~\emptyset}
        \1,\3 (v \neq 0)
$$

$$
\textbf{(loop 1)} \5
    \frac
    { \textbf{(if 1)} \3
    (if~mem(1)~...) \xrightarrow{0} break \1\nearrow^{~\emptyset} }
%   -----------------------------------------------------------
    { (if~mem(1)~...~@~loop~q) \xrightarrow{0} (break~@~loop~(if~...))
        \1\nearrow^{~\emptyset} }
$$

$$
\textbf{(loop 1)} \5
    \frac
    { \textbf{(if 0)} \3
    (if~mem(0)~...) \xrightarrow{0} await~A \1\nearrow^{~\emptyset} }
%   -----------------------------------------------------------
    { (if~mem(0)~...~@~loop~q) \xrightarrow{0} (await~A~@~loop~(if~...))
        \1\nearrow^{~\emptyset} }
$$

$$
\textbf{(loop 3)} \3
    (break~@~loop~(if~...)) \xrightarrow{0} nop \1\nearrow^{~\emptyset}
$$


$$
\textbf{(loop 2)} \3
    \frac
    { \textbf{(await 1)} \3 await~A \xrightarrow{(A,0)} delay~0 }
%   -----------------------------------------------------------
    { await~A~@~loop~(if~...) \xrightarrow{(e,i)} delay~0~@~loop~(if~...) }
$$

$$
\textbf{(loop 1)} \5
    \frac
    { \textbf{(delay)} \5 (delay~0) \xrightarrow{0} nop
        \1\nearrow^{~\emptyset} }
%   -----------------------------------------------------------
    { delay~0~@~loop~(if~...) \xrightarrow{0} nop~@~loop~(if~...)
        \1\nearrow^{~\emptyset} }
$$

$$
\textbf{(loop 2)} \3
    (nop~@~loop~(if~...) \xrightarrow{0}
        (if~...)~@~loop~(if~...) \1\nearrow^{~\emptyset}
$$

$$
\textbf{(or)} \5
    \frac
    { \textbf{(seq)} \1 ... \xrightarrow{(A,0)} ... \5
        \textbf{(or)} \3
            \frac
            { \textbf{(await~2)} \1 await~b \xrightarrow{(A,0)} await~b \5
                \textbf{(seq)} \1
                    \frac
                    { \textbf{(await~1)} \1 await~A \xrightarrow{(A,0)} delay~0 
}
                %   -----------------------------------------------------------
                    { (await~A~;~emit~a) \xrightarrow{(A,0)} (delay~0~;~emit~a) 
            }
            }
        %   -----------------------------------------------------------
            { (await~b~or~(await~A;emit~a)) \xrightarrow{(A,0)}
                (await~b~or~(delay~0;emit~a)) }
    }
%   -----------------------------------------------------------
    { (...~or~(await~b~or~(await~A;emit~a))) \xrightarrow{(A,0)}
        (...~or~(await~b~or~(delay~0;emit~a))) }
$$

$$
\textbf{(or)} \5
    \frac
    { p \xrightarrow{(e,i)} p' \5 q \xrightarrow{(e,i)} q' }
%   -----------------------------------------------------------
    { (p~or~q) \xrightarrow{(e,i)} (p'~or~q') }
$$

$$
\textbf{(or 2)} \3
    \frac
    { isReady(q,i) \5 q \xrightarrow{i} q' \1\nearrow^{~(e,i)} }
%   -----------------------------------------------------------
    { (...~or~(await~b~or~(delay~0~;~emit~a))) \xrightarrow{0}
        (...~or~nop~;~emit~a) \1\nearrow^{~(e,i)} }
$$

$$
    \frac
    {
        \frac
        {
            \textbf{(seq 1)} \1 \frac {
                \textbf{(delay)} \1 (delay~0) \xrightarrow{0} nop
            }{
                (delay~0;~emit~a) \xrightarrow{0} nop~;~emit~a
            }
        \3
            \textbf{(seq 2)} \1 (nop~;~emit~a) \xrightarrow{0} emit~a
        }
%       -----------------------------------------------------------
        { (delay~0~;~emit~a) \xrightarrow{0} emit~a }
    \3
        \textbf{(emit)} \1 emit~a \xrightarrow{0} delay~1 \1\nearrow^{~(a,2)}
    }
%   -----------------------------------------------------------
    { (delay~0~;~emit~a) \xrightarrow{0} delay~1 \1\nearrow^{~(a,2)} }
$$

(seq 3)
loop do
    par/or do
        se1 ;
        break;
    with
        se2 ;
    end
    se3;
end

- restricao de tight loop é a única na sintaxe
- depois vou fazer a semantica para ND
    - criar um grafo
    - mostrar que é o unico grafo
- com a semantica de ND, também posso "falhar" em caso de tight loop, o que 
  removeria a restricao na sintaxe e daria semantica p/ o tight loop

Prova 1: mostrar que a inclusão de $(f,i+1)$ e $(\_,i+2)$ em E-global na regra 
(emit) não causa problemas, já que as regras em paralelo estão reagindo a "i" e 
a reação a "i+1" só é possível quando todas as outras terminarem (via regra 
step)

Prova 2: provar que eu chego em $S_{\emptyset}$

Prova 3: não é possível fazer infinitas transições
- preciso remover "same" e "nop"

Prova 1 e 2 -> Reaction chain executa em bounded time

\begin{figure}[t]
\begin{eqnarray*}
& mem \xrightarrow{i} nop(v)
    & \textbf{(mem)}        \\
%%%
& emit~e \xrightarrow{i} delay~(e,i+1)
    & \textbf{(emit)}       \\
%%%
& \frac
    { mem \xrightarrow{i} nop(v) \1,\3 (v \neq 0) }
% -----------------------------------------------------------
    { (if~mem~then~p~else~q) \xrightarrow{i} p }
    & \textbf{(if-true)}    \\
%%%
& \frac
    { mem \xrightarrow{i} nop(0) }
% -----------------------------------------------------------
    { (if~mem~then~p~else~q) \xrightarrow{i} q }
    & \textbf{(if-false)}   \\
%%%
& (if~mem~then~p) \xrightarrow{i} (if~mem~then~p~else~nop)
    & \textbf{(if-else)}    \\
%%%
& \frac
    { p \xrightarrow{i} p' }
%   -----------------------------------------------------------
    { (p~;~q) \xrightarrow{i} (p'~;~q) }
    & \textbf{(seq-1)}      \\
%%%
& (nop~;~q) \xrightarrow{i} q
    & \textbf{(seq-2)}      \\
%%%
& (break~;~q) \xrightarrow{i} break
    & \textbf{(seq-3)}      \\
%%%
& { (loop~p) \xrightarrow{i} (p~@~loop~p) }
    & \textbf{(loop-exp)}   \\
%%%
& \frac
    { p \xrightarrow{i} p' }
% -----------------------------------------------------------
    { (p~@~loop~q) \xrightarrow{i} (p'~@~loop~q) }
    & \textbf{(loop-1)}    \\
%%%
& (nop~@~loop~p) \xrightarrow{i} loop~p
    & \textbf{(loop-2)}    \\
%%%
& (break~@~loop~p) \xrightarrow{i} nop
    & \textbf{(loop-3)}     \\
%%%
& \frac
    { p \xrightarrow{i} p' }
%   -----------------------------------------------------------
    { (p~and~q) \xrightarrow{i} (p'~and~q) }
    & \textbf{(and-1)}      \\
%%%
& \frac
    { q \xrightarrow{i} q' }
%   -----------------------------------------------------------
    { (p~and~q) \xrightarrow{i} (p~and~q') }
    & \textbf{(and-2)}      \\
%%%
& (nop~and~q) \xrightarrow{i} q
    & \textbf{(and-3)}      \\
%%%
& (p~and~nop) \xrightarrow{i} p
    & \textbf{(and-4)}      \\
%%%
& \frac
    { q=break \1\vee\1 isBlocked(q) }
%   -----------------------------------------------------------
    { (break~and~q) \xrightarrow{i} break }
    & \textbf{(and-5)}      \\
%%%
& \frac
    { p=break \1\vee\1 isBlocked(p) }
%   -----------------------------------------------------------
    { (p~and~break) \xrightarrow{i} break }
    & \textbf{(and-6)}      \\
%%%
& \frac
    { p \xrightarrow{i} p' }
%   -----------------------------------------------------------
    { (p~or~q) \xrightarrow{i} (p'~or~q) }
    & \textbf{(or-1)}   \\
%%%
& \frac
    { q \xrightarrow{i} q' }
%   -----------------------------------------------------------
    { (p~or~q) \xrightarrow{i} (p~or~q') }
    & \textbf{(or-2)}   \\
%%%
& \frac
    { q=nop \1\vee\1 isBlocked(q) }
%   -----------------------------------------------------------
    { (nop~or~q) \xrightarrow{i} nop }
    & \textbf{(or-3)}   \\
%%%
& \frac
    { p=nop \1\vee\1 isBlocked(p) }
%   -----------------------------------------------------------
    { (p~or~nop) \xrightarrow{i} nop }
    & \textbf{(or-4)}   \\
%%%
& \frac
    { q=nop \1\vee\1 q=break \1\vee\1 isBlocked(q) }
%   -----------------------------------------------------------
    { (break~or~q) \xrightarrow{i} break }
    & \textbf{(or-5)}   \\
%%%
& \frac
    { p=nop \1\vee\1 p=break \1\vee\1 isBlocked(p) }
%   -----------------------------------------------------------
    { (p~or~break) \xrightarrow{i} break }
    & \textbf{(or-6)}   \\
\end{eqnarray*}%
\caption{ Small-step rules. }
\label{lst:ceu:1}
\end{figure}

\end{document}
\end{comment}
