We propose a new abstraction mechanism for a synchronous reactive 
programming language: Organisms reconcile objects and threads into a single 
concept.
We enforce lexical scope for dynamic instances and also restrict the use of 
explicit references to organisms.
This way, we can eliminate typical issues in dynamic allocation, such as 
memory leaks, danging pointers, and garbage collection.

\documentclass[11pt,a4paper]{article}
\usepackage[nohead,a4paper,lmargin=1.5cm,rmargin=2.0cm,tmargin=2.0cm,bmargin=2.0cm]{geometry}

\usepackage{verbatim}
\usepackage{xspace}
\usepackage{url}

\newcommand{\CEU}{\textsc{C\'{e}u}\xspace}
\newcommand{\code}[1] {{\small{\texttt{#1}}}}

\title{
    Imperative Reactive Programming with \CEU
}
%\author{Francisco Sant'Anna}

\begin{document}

\maketitle

% wc = 19 19 394

%\abstract{
\section{Abstract}
The origins of reactive programming date back to the early 80's with the 
co-development of two styles of synchronous languages:
%
The imperative style of Esterel organizes programs with control flow 
primitives, such as sequences, repetitions, and also parallelism.
%
The dataflow style of Lustre represents programs as graphs of values, in which 
a change to a node updates its dependencies automatically.

In recent years, Functional Reactive Programming modernized the dataflow style 
and became mainstream, producing a number of languages and libraries, such as
Flapjax, Rx (from Microsoft), React (from Facebook), and Elm.
%
In contrast, the imperative style did not follow this trend and is now confined 
to the domain of real-time embedded control systems.
% such as flight control on avionics and anti-collision equipment on automobiles.

We present the programming language \CEU, a contemporary outlook of imperative 
reactivity that aims to expand the application domain of this family with new 
abstraction mechanisms.
%which supports a rich set of control abstractions and static safety 
%guarantees.
%
\CEU has its roots on Esterel and relies on a similar synchronous and 
deterministic execution model that simplifies the reasoning about concurrency 
aspects.
%
%In particular, the disciplined stepwise synchronous execution of programs 
%allows for concurrent lines to safely share variables.
%
The example below in \CEU prints the ``Hello World!'' message every second, 
terminating with a key press:

\begin{verbatim}
    par/or do
       every 1s do
          print("Hello World!");
       end
    with
       await KEY;
    end
\end{verbatim}

The use of control primitives, such as \code{await}, \code{every}, and 
\code{par/or}, hides most complexity related to flows of execution and their 
life cycles (e.g., starting, aborting, and resuming them).
%
In particular, the \code{await} statement is the most representative of \CEU, 
capturing the imperative and reactive nature of the language at the same time.
%
Awaiting multiple events in parallel releases programmers from the infamous 
\emph{callback hell}, which is often a concern in general purpose imperative 
languages.

The new abstraction mechanisms of \CEU to be presented expands classical object 
orientation with instances that are reactive to the environment.
%
They make imperative reactivity suitable for applications outside the embedded 
domain.
%
As an example, we implemented in \CEU a game of moderate complexity for 
Android.
%(approximately 5,000 lines of code).
%
The imperative reactive style of \CEU is an effective alternative to Functional 
Reactive Programming that complements the realm of synchronous programming.

\begin{comment}
programs close to specification

adv: convenience of imperative w/o the burden
also solves the callback hell
- big gap for imperative reactivity

React (vs FRP)
http://www.debjitbiswas.com/genuinely-functional-user-interfaces-and-react/

 170   541  6207 controllers.ceu
  20    51   518 fnts.ceu
 476  1765 18627 main.ceu
 575  1828 17676 objs.ceu
  33    94   918 points.ceu
  45   120  1351 snds.ceu
  94   254  2971 texs.ceu
1413  4653 48268 total
\end{comment}

%}

\newpage

\section{Comments}

\begin{itemize}
    \item Presentation Outline:
        \begin{itemize}
        \item Reactive applications: characteristics, challenges
        \item Reactive models: Declarative/Dataflow/FRP vs Control/Imperative
        \item Overview of \CEU
        \item Reactive ``Hello World!''
        \item Reactive abstractions in \CEU
        \item Demos: aprox. 10 incremental live demonstrations
        \end{itemize}

    \item Intended audience:
        \begin{itemize}
        \item Programmers interested in real-time and reactive concurrency 
      (embedded systems, games, and multimedia).
        \item Requires knowledge  of C or C++.
        \end{itemize}

    \item Previous StrangeLoops?
        \begin{itemize}
        \item No, it would my first time.
        \end{itemize}

    \item Speaking Experience:
        \begin{itemize}
        \item Talks related to \CEU:
            \begin{itemize}
            \item 2014 - Arduino Day - Presentation
               \url{(http://thesynchronousblog.wordpress.com/2014/03/31/ceu-on-arduino-day/)}
            \item 2013 - SenSys Conference - Full Paper
               \url{(http://sensys.acm.org/2013/index.html)}
            \item 2013 - REM Workshop (inside OOPSLA) - Full Paper
               \url{(http://soft.vub.ac.be/REM13/)}
            \item 2011 - SenSys Conference - Doctoral Colloquium
               \url{(http://www.cse.ust.hk/~lingu/SenSys11DC/)}
            \end{itemize}
        \item I'm also a CS teacher and we've been using \CEU in Distributed 
System classes.
        \end{itemize}

    \item Maturity of \CEU:
        \begin{itemize}
        \item Developed over the past 6 years in a PhD thesis (mine).
                  \url{(http://www.ceu-lang.org/)}
        \item Open sourced last year.
                  \url{(http://github.com/fsantanna/ceu/)}
        \item Participating this year with one project in the ``Google Summer 
              of Code'' (under the umbrella of the LabLua at PUC-Rio).
                  \url{(http://www.lua.inf.puc-rio.br/gsoc/ideas2014.html)}
        \item Evaluated in the context of Wireless Sensor Networks together 
          with the DCS group at Chalmers University (Sweden).
                  \url{(http://www.cse.chalmers.se/research/group/dcs/)}
        \item Used by other graduate and undergraduate students.
        \item No sensible uses outside of our domains (as far as we know).
        \end{itemize}
\end{itemize}

\end{document}
