\section{Evaluation}
\label{sec.evaluation}

to be done

\begin{comment}

ROM RAM loc tok bytes gzipped

quantitative
memory, expressiveness

qualitative
battery, safety, expressiveness, responsiveness

Similar evaluation in a previous work \cite{wsn.sol}.

We make extensive use of macros.
\CEU{} has a good composibility XXX.

\footnote{
In order to adapt, as they are imported with a single \texttt{include} macro 
call.
Shared without any changes or specific copies into the source, just a call to 
the \texttt{include} macro of the \emph{m4 processor}.
}

(~A || ~B || ~C || ~D)
qual o tamanho disso??
automatic bookkeeping (creation/start/destroy)/ stack / shceduling
in one line, create/spawn 4 threads, once any of them terminate, kill the 
others
expressive and also efficient, theres no int representation of a "thread" with 
an structure with state in CEU
just set flags/gates

coarse grained vs fine grained multithreading

auto book allows fine-graining

ceu is stackless

also syntactic support

SOL, heavy syntax, what about additional space(stack)

falar de composibility

comparacao como artigo HIGH
incluir linhas de codigo

bateria, uso de memoria igual, a nao ser que o programa perca tempo em coisas 
que os outros nao fazem
em relacao a threads, no context switch
mesmo para os asyncs nao ha context switch
no run-time overheads due to threads creation/destroy, just set flags

bateria / memory ==> code size
expressiveness   ==> parallel compositions, FRP - lines of code
responsiveness   ==> no loops, asyncs
safety           ==> temporal analysis, no memory alloc, no stack, determinism, 
no loops

memory / total / ram

battery

processing
\end{comment}

\section{Related Work}
\label{sec.related}

Our work is strongly influenced by the Esterel language \cite{esterel.design}.
In Esterel, time is defined as discrete reaction steps in which multiple 
external signals (events in \CEU) can be queried on their presence status.
This semantics has more proximity with that of electronic circuits, but 
simultaneous events inhibits the temporal analysis of \CEU.
For instance, variable manipulation in Esterel is restricted to a single 
process (trail in \CEU).

%- no support for internal events
%- no simulation, no async in Esterel
%- no physical time, semantic tighted to the rest of the language (no first 
%class)
%esterel sample-driven vs event-driven

The \emph{Functional Reactive Programming (FRP)} paradigm brings dataflow 
behavior to functional languages \cite{frp.principles}.
Our implementation borrowed some ideas from FrTime \cite{frp.embedding}, such 
as the push-driven evaluation, glitch prevention, and an insight to avoid tight 
cycles and preserve determinism (not covered in this paper).
However, the dynamic nature of FRP does not match the efficiency and safety 
requirements of WSNs.

to be done

\begin{comment}

%Although essentially functional, FRP systems must be adapted to stateful 
%libraries \cite{frtime.crossing}, e.g., GUI toolkits, embedded systems, in 
%order to interact with the real world.
%However, FRP lacks , probably due its dynamic nature.

Regarding languages specific for WSNs, the event-driven nesC \cite{wsn.nesc} 
provides a simple yet flexible concurrent model featuring detection of possible 
race conditions at compile time.
However, this detection is limited to asynchronous code.
Two synchronous tasks that react to the same event and access the same data do 
not characterize a race condition, as they are always serialized.
This serialization is often arbitrary (e.g. respecting the order they are 
scheduled) and might hide non-deterministic programs.

TinyThread supports both, but mantaining the models, that is, additional 
complexity, sum of both.
not only both are less expressive, but also diff phil.

if a system supports both
differents in philosophy
diff in impl.
do not match a single syntax
are not tought together

Finally, SOL \cite{} is a Esterel-like language targeting WSNs.
Also, Table~\ref{} and we show equivalent results for ROM and RAM.

nondet for Pars/raises

%Some works, external tools that infers the amount of memory.
%In event-driven programming it is common to xxx state on the heap, and no 
%warranty about enough space and correct management is supported.
%Multithreading uses the stack, being better about leaks, however again, no 
%warranty about stack overflows is given.
%Also, if the creation of threads is dynamic, then not enough memory error.

%related.
%protothreads no stack space
    %Protothreads \cite{wsn.protothreads} provides stackless cooperative 
    %concurrency, but lacks local variables support.

% related work do contiki

more complex: proto vs threads.

coarse grained multitasking.
manual preemption.
no syntax support (function calls);

Finally, except for our previous work with LuaGravity \cite{luagravity.sblp} 
(which is out of scope for WSNs), we have no knowledge of other attempts to 
unify the control \& dataflow styles of synchronous languages.

\end{comment}

\section{Conclusion}
\label{sec.conclsion}

to be done

\begin{comment}
SAFENESS
in ceu the programmer cannot write synchronous bad code as it cannot write a 
loop w/o an await
hence the code must be async and will run with low priority

We presented \CEU{}, a language that aims to unify the apparently antagonistic 
imperative and declarative reactive programming models.
\CEU{} is based on a small synchronous kernel similar to Esterel that provides 
reaction to events and imperative constructs such as sequences, parallel 
blocks, and assignments.

In order to support FRP in \CEU{}, we added \emph{internal events} as a 
communication mechanism among \CEU's trails.
Internal events follow a stack execution policy, forcing that a sequence of 
triggers reacts backwards.
This semantics is required to respect an arbitrary level of dependency among 
behaviors, and also avoids cyclic dependencies for mutually dependent 
behaviors.

The coexistence of concurrency (through parallel blocks) and global assignments 
requires a rigorous temporal analysis in programs to detect any possible kind 
of non-determinism.
\CEU{} transforms programs into deterministic finite automatons to detect the 
three kinds of non-determinism that can arise in programs: concurrent access to 
variables, par/or terminations and loop escapes.

Besides supporting both imperative and declarative reactive programming models, 
\CEU{} also brings novel features to the scene: native support for 
\emph{physical time} and simulation from within the own programs.

phys time also SAFE+express

% TODO: claim recurrent use justifies
We believe that the recurrent use of physical time in reactive applications 
already justifies providing a convenient syntax for timing purposes.
Furthermore, native support is also \emph{desired} to avoid dealing explicitly 
with \emph{residual delta times} from expired timers; and is \emph{required} to 
allow extending \CEU's temporal analysis to include physical time.

% TODO: claim
By allowing asynchronous blocks to enqueue inputs to the synchronous side, it
is easy to simulate and test the execution of \CEU{} programs with total 
control and accuracy with regard to the order of input events and passage of 
time: all is done with the same language and inside the own programs.

By providing ways to simulate programs in the own language, \CEU{} becomes 
self-sustaining, not depending on any environment to run its programs.
Simulation is an important aspect for cross-compiling development as in WSNs, 
where software simulators are usually adopted and are never accurate, nor 
platform independent.

% TODO: application
%We have been using \CEU{} to develop applications for wireless sensor 
%networks.

% TODO: future work

We believe that, safety and predictability is more important than flexibility 
in the context of WSNs applications.
Robust operation.

emerging multi-core, good field for ceu, as asyncs do not share data

It is interesting to note that, although apparently antagonistic, both 
imperative and dataflow styles are, at the lowest-level, grounded on a TODO 
synchronous event-driven subsystem.
In other words, event-driven programming can be used as an implementation 
technique for both styles \cite{my.luagravity}.
\end{comment}

\begin{comment}
\section{Runtime \& Implementation}
\label{sec.runtime}

- VM

    * altura / glitches / prios
    * break or, falar em exec model
    * TODO: altura
      the runtime must always respect sequentiality.
      just as the runtime respects the order of subexpression in a seq block, 
xxx.

Runs to completion, any order, even with interleave.
\CEU{} is by default fully deterministic (but see Section~\ref{det} and 
\ref{nondet}).
The only requirement is that no other event is handled in the meantime.

Although most of \CEU's expressions have a clear semantics by themselves, XXX. 

TODO:PAROR prop chain, nd.
If two or more subexpressions terminate during the same time unit (see TODO), 
the SplitOr block returns \texttt{nd}, a s (even if they return the same 
value).
If multiple events happen during the same propagation chain, the expression 
returns $nd$.

This way, the order in which the lines of execution run is irrelevant to the 
XXX of the program.
Possible to create safe deterministic programs when appropriate
qualquer ordem, interleave.
nao so eh equivalente a sequencial, como a implementacao pode usar isso (single 
core).

Given the same input configura-
tion, the system will always yield the same output.


non-det (see extensions)

\begin{Verbatim}[commandchars=\\\{\}]
{ {!a;$v=1} , {!a;!b;$v=1} }
\end{Verbatim}

glitches

Another fundamental characteristic of asynchronous systems is their
inherent non-deterministic behavior [Lee]. From the perspective of a CPU,
the composition of concurrent processes may be seen as the interleaving of
their statements in a single process. First, it is impossible to predict the
order in which statements are interleaved, as process scheduling timing is non-
deterministic. Second, for the same reason, each time a concurrent system
restarts, a different scheduling probably takes place. The lack of enforced
synchronization between processes makes impossible to ensure a deterministic
behavior.
This characteristic leads to difficulty in achieving global consensus
over information, as data obtained by the receiving end can reflect a past
state in the sending end. Delayed communication also introduces the need for
buffers, yet another concern to deal with (e.g. overflow and underflow).

-- DIFFERENCES TO MSG-BASED SYSTEMS
Two fundamental characteristics about events differ \CEU{} from message-passing 
systems such as XXX and YYY:
First, \emph{events are not buffered}---if an event occurs and no expression is 
waiting for it, the event is simply lost.
Second, \emph{events are broadcast to the whole program}---if $N$ expressions 
in parallel are waiting for the same event, they are all awaken when the event 
happens.
-- synchronous

\xxx Documentar difs entre ext/int
TODO: why?
 multiple at the same time, externals need an order
 extensible control-flow: exceptions
 ({ !A=1; !A })*   -- tb explica porque I/O sao separados

int - para controle, spawn/nao trigg, igual a var, apenas um valor, aguarda 
prioridade/nondet/executa uma vez, sem idx

ext - para interacao, nao spawn/trig, nao guarda vals, n valores, in/out com 
arity diffs, nao aguarda prio/nondet/execut n vezes, nao tem spawn, um evt pode 
ser somente in ou out, mas nao os dois, area reservada p/ await, o trig eh 
pelos regs correntes mesmo.
so preciso de uma area res para await pois dois nao podem acontecer ao mesmo 
tempo

se um ext fosse de in e out ao mesmo tempo, precisaria dos spawns, exatamente 
como no internal.
assim nao seria possivel dois prints no mesmo ciclo, ja que teria que esperar 
na mesma prioridade e apenas dar um trigger.

se o ext tivesse spawn ele entraria no modo sincrono, lembrar que ele eh o 
tempo externo, preciso disso (pq?)

- extending the langugage
    - macros
    - composibility
    - local blocks

\section{Programming Techniques}

exemplo grande para ver quantas "threads" coexistiriam

state machines

buffers

composibility -> macros
different from macroing thread declarations
suggests low cost

\section{Implementation}

high-concurrency furar fila c/ prios para eventos
async sempre menor, timers, eventos

static memory allocation

low cost per trail

Implementation:
Ceu has no limit on the number of "threads", and at a given time it uses 
exactly the size needed for that.
Not stack, not heap, its just static memory "register" known at compile time.
Mostrar um exemplo em que duas threads estao em sequencia e usam o mesmo 
espaco.

\end{comment}

\bibliographystyle{acm}
\bibliography{other}

\end{document}
