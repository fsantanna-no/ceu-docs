\documentclass[11pt,a4paper]{report}
\usepackage[nohead,a4paper,lmargin=1.5cm,
rmargin=2.0cm,tmargin=2.0cm,bmargin=2.0cm]{geometry}

\usepackage[brazilian]{babel}
\usepackage[utf8]{inputenc}
\usepackage[T1]{fontenc}

\usepackage[pdftex]{graphicx}
\usepackage{verbatim}
\usepackage{color}
\usepackage{url}

\newcommand{\TODO}[1]{\colorbox{red}{TODO: #1}\linebreak}

\newcommand{\CEU}{\textsc{C\'{e}u}}

\title{Plano de Trabalho}
\author{Francisco Sant'Anna}
\date{\today}

\begin{document}

\maketitle

\tableofcontents

\chapter{Resumo}

Este documento apresenta o plano de trabalho a ser desenvolvido durante o
Doutorado Sanduíche na Universidade de Standford, nos Estados Unidos.
O plano de trabalho é um dos requisitos para a candidatura a uma bolsa no 
programa ``Ciência sem Fronteiras'' do governo brasileiro.

O objetivo do projeto de Doutorado Sanduíche é unir os grupos na PUC-Rio e 
Stanford, ambos com pesquisa em Redes de Sensores sem Fio, a fim de aplicar a 
pesquisa sendo desenvolvida no doutrado aos projetos do grupo em Stanford.

As Redes de Sensores sem Fio (RSSF) são uma peça importante para viabilizar o 
tem se chamado de \emph{internet das coisas}, onde tudo, desde XXX a YYY estão 
conectados por uma rede sem fio, de baixo consumo comunicam um novo XXX.
Dentre algumas das aplicações, podemos citar o usa de RSSF para monitoramento 
de pacientes, prevenção de desastres ambientais, implantação de cidades 
inteligentes (ex. trânsito, iluminação).

O pesquisa de doutorado se enquandra em uma das áreas prioritárias do programa 
``Ciência sem Fronteiras'': \emph{Computação e Tecnologias da Informação}.

\chapter{Informações Gerais}

Apresentamos aqui as pessoas e instituições ligadas à proposta: universidade no 
Brasil, universidade no exterior, aluno, orientadores no Brasil e orientador no 
exterior.

\section{Universidade/Programa no Brasil}

O Departamento de Informática da PUC-Rio teve o primeiro curso de mestrado do 
Brasil (1968) e segundo de doutorado (1975) na área.
Já titulou mais de 1000 alunos entre mestres e doutores.
Sempre obteve da CAPES o conceito máximo concedido a programas da área de 
Ciência da Computação, tendo nota 7 desde 2004.

\section{Universidade/Programa no Exterior}

A Universidade de Stanford é referência mundial na área de Ciência de 
Computação.
Em 2011, foi considerada a segunda melhor universidade do mundo em Ciência da 
Computação em 2011 por um site especializado%
\footnote{\url{http://www.topuniversities.com/university-rankings/world-university-rankings/2011/subject-rankings/engineering/computer-science}}.
Pelo seu quadro de pesquisadores já passaram 18 prêmios Turing (o prêmio máximo 
para um pesquisador da área).

O laboratório \emph{Stanford Information Networks Group (SING)}, onde atua o 
orientador no exterior, pesquisa protocolos e arquiteturas de redes, com foco 
em redes de sensores de baixo consumo.

Homepage do Laboratório:
\url{http://sing.stanford.edu/}

Lista de Publicações:
\url{http://sing.stanford.edu/singpubs.html}


\section{Aluno candidato}

\emph{Francisco Figueiredo Goytacaz Sant'Anna} é bolsista CNPq de doutorado do 
Departamento de Informática da PUC-Rio desde agosto de 2009, com previsão de 
término para o segundo semestre de 2013.
%Foi o primeiro colocado no exame de admissão de doutorado.
Seu mestrado também foi realizado no mesmo departamento, entre 2007 e 2009.
É formado em Engenharia da Computação, também pela PUC-Rio.

Participou da especificação e implementação do padrão da TV Digital brasileira 
\cite{} no Laboratório Telemídia da PUC-Rio (que desenvolveu o padrão), sendo 
bolsista do projeto entre 2007 e 2009.

O título provisório de sua tese de doutorado é
\emph{"Safe and High-level Programming for Embedded Systems"}.

Como parte do doutorado também está desenvolvendo a linguagem \emph{Céu}%
\footnote{\url{http://www.ceu-lang.org/}},
que põe em prática as idéias e conceitos desenvolvidos durante a pesquisa.

Homepage: \url{http://www.lua.inf.puc-rio.br/~francisco/}

Currículo Lattes: \url{http://lattes.cnpq.br/0077491494754494}

\section{Orientador no Brasil}

\emph{Roberto Ierusalimschy} é professor associado do Departamento de 
Informática da PUC-Rio, onde atua na área de Linguagens de Programação.
Obteve o doutorado no Departamento de Informática da PUC-Rio, e pós-doutorado 
na University of Waterloo, no Canadá.
É bolsista de Produtividade em Desenenvolvimento Tecnológico 1C.

Roberto também é o principal projetista da linguagem de programação Lua e autor 
do livro ``Programming in Lua'' \cite{}.
Lua é mundialmente reconhecida como uma das principais linguagem de script para 
video games%
\footnote{\url{http://www.gdmag.com/blog/2012/01/front-line-award-winners.php}}.

Entre os meses de janeiro a março de 2012, Roberto está como professor 
visitante na Universidade de Stanford para ministr o curso de inverno 
\emph{Scripting Embedded Systems with Lua}%
\footnote{\url{http://las.stanford.edu/group/las/cgi-bin/drupal/people/tinker-professors/roberto-ierusalimschy}}.
Tanto a universidade (Stanford), quanto o professor em contato direto (Philip 
Levis), são os mesmos apresentados nesta proposta.

Webpage: \url{http://www.inf.puc-rio.br/~roberto/}

Currículo Lattes: \url{http://lattes.cnpq.br/0427692772445368}

\section{Co-orientadora no Brasil}

\emph{Noemi de La Roque Rodrigues} é professora associada do Departamento de 
Informática da PUC-Rio, onde atua na área de sistemas concorrentes e 
distribuídos.
Obteve o doutorado no Departamento de Informática da PUC-Rio, e pós-doutorado 
na University of Illinois at Urbana-Champaign, nos Estados Unidos.

Atualmente coordena o projeto Projeto de Cidades do XXX, em parceria com mais 
de XX instituições de pesquisa brasileiras, que visa ZZZ.

Webpage: \url{http://www.inf.puc-rio.br/~noemi/}

Currículo Lattes: \url{http://lattes.cnpq.br/4933326132948063}

\section{Orientador no Exterior}

\chapter{Objetivos}

Título: ``''

Área:

Relevância:

O grupo da PUC-Rio atua no suporte de linguagens.
A linguagem Céu, desenvolvida como para do doutorado, é destinada a desenvolver 
aplicações para sistemas embarcado, tais como encontrados nas RSSF.

O grupo de Stanford tem reconhecida \emph{expertise} no desenvolvimento de 
novos algoritmos e aplicações para RSSF.

A idéia é aplicar a linguagem aos projetos existentes e avaliar os ganhos com a 
nossa abordagem.

Aplicações reais, em .

\chapter{Metodologia}

Memória ROM/RAM, consumo de energia, linhas de código, detecção de bugs.

\bibliographystyle{acm}
\bibliography{other}

\end{document}
